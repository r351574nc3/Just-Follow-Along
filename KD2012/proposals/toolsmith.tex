\documentclass[12pt,notitlepage]{article}
\author{Leo Przybylski}
\usepackage{fancyhdr}
\usepackage{graphicx}
\title{\includegraphics[width=0.75\textwidth]{kuali_base.png}\\For the Right Tool, See a Smith}
\date{}
\pagestyle{fancy}
\fancyhead{} % Clear all header fields 
\fancyhead[OL]{\sectionmark}
\fancyhead[OR]{\includegraphics[height=26pt]{kuali_base.png}}% 
\fancyhead[ER]{\sectionmark}
\fancyhead[EL]{\includegraphics[height=26pt]{kuali_base.png}}% 

\begin{document}
\maketitle
\abstract{I once attended a meeting at UA debriefing some attendies of
  a Higher Education User Group (HEUG) conference. My biggest
  take-away was that institutions were willing to share what they were
  doing and the troubles they were having, but not how they were doing
  things or how they solved any of their problems. A lot of institutions
  suffer many of the same issues.

  This is one area Kuali has really accelled as a community. This
  session is the opporunity to share your tools.}

\section{Objectives}

Much of the time during software development we encounter
  scenarios that make it cumbersome for regular tasks. Kuali is not an
exclusion here. Even different process or business models not fitted
for your Kuali application will put extra strain on your business in
general if the right tools do not exist. 

One example is in change
management. Determining changes for a release is really not part of
the delivered infrastructure for Kuali applications. Further, requirements can
vary from institution to institution. Sometimes this process can be
lengthy and frought with human error. The best way to mitigate human
error is with tools.

Another example is environment triage. Institutions probably vary here
as well. The process for triage and requirements on quality can very
at institutions. This often means the right tool is different at
institutions.

This session is about discovering what institutions are doing for
processes, what kinds of tools they are using to overcome overhead and 
simplify quality assurance, what kinds of tools are used for
automation,
 and learning about who develops these
tools at institutions. Do institutions employ much devops for their
tools? Do they rely on tools other institutions use? Do they just not
bother and live with the manual overhead?


\begin{itemize}
\item Participants will discuss about custom processes to improve
  development.
\item To discover from other institutions the kinds of tools they use
  to improve processes and the quality of their development.
\item To discover who develops custom tools for institutions.
  \begin{itemize}
    \item Learn about the impact of DEVOPS at institutions.
  \end{itemize}
\end{itemize}

\section{Audience}
Technical track for
\begin{itemize}
\item DEVOPS
\item Developers
\item Development Managers
\end{itemize}
\end{document}
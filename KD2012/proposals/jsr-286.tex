\documentclass[12pt,notitlepage]{article}
\author{Leo Przybylski}
\usepackage{fancyhdr}
\usepackage{graphicx}
\title{\includegraphics[width=0.75\textwidth]{kuali_base.png}\\The Status is Not Quo! JSR-286 and Rice}
\date{}
\pagestyle{fancy}
\fancyhead{} % Clear all header fields 
\fancyhead[OL]{\sectionmark}
\fancyhead[OR]{\includegraphics[height=26pt]{kuali_base.png}}% 
\fancyhead[ER]{\sectionmark}
\fancyhead[EL]{\includegraphics[height=26pt]{kuali_base.png}}% 

\begin{document}
\maketitle
\abstract{In this session, Leo will go through various portal customizations. This includes customizing the portal via KIM and adding various usability features like help and tooltips.

This session will also review JSR-286. JSR-286 is the 2.0 portlet specification. This is a walkthrough of implementing web services as a portlet.}
\section{Objectives}
\begin{itemize}
\item Developers will see what goes into producing a portlet implementation
\item Developers will learn/understand how to create a web services client in java to communicate with Kuali Rice
\item Developers will learn how to deploy a portlet to a portlet
  container
\end{itemize}

\section{Presenters}
Leo Przybylski

\section{Audience}
Technical track for
\begin{itemize}
\item developers
\end{itemize}


\subsection{Level}
Intermediate

\section{Focus Area}
Technical

\section{Track}
Kuali Financial System

\section{Format}
Information Session

\end{document}
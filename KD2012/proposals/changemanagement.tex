\documentclass[12pt,notitlepage]{article}
\author{Leo Przybylski}
\usepackage{fancyhdr}
\usepackage{graphicx}
\title{\includegraphics[width=0.75\textwidth]{kuali_base.png}\\Maven
  Overlays: Use Cases for Change Management}
\date{}
\pagestyle{fancy}
\fancyhead{} % Clear all header fields 
\fancyhead[OL]{\sectionmark}
\fancyhead[OR]{\includegraphics[height=26pt]{kuali_base.png}}% 
\fancyhead[ER]{\sectionmark}
\fancyhead[EL]{\includegraphics[height=26pt]{kuali_base.png}}% 

\begin{document}
\maketitle
\abstract{Changes can come from numerous places. They can come from
  the Kuali Foundation,vendors, consultants, patches, and
  developers. Two common problems with these for institutions are
  determining how to stay on top of them and how projects should be
  organized to deal with incoming changes as well as local
  customizations. In this session, change managers and developers will
be introduced to processes and policies for managing incoming changes
from varying sources. Devops will observe how to leverage maven
overlays and VCS (SVN and Git) to structure your projects. We will
observe caveats and exceptional cases as well as common cases like
dealing with distributing Rice updates to KFS or KC instances. We will
examine maven overlay use cases for Rice, KC, and KFS projects.}

\section{Objectives}
\begin{itemize}
\item Describe how to setup maven overlays for kuali projects.
\item Show how to use maven overlays in development as well as for packaging
  individual changes for change promotion.
\item Describe how to utilize features in SVN and/or Git for managing
  changes and promoting them to environments.
\item Show the advantages of convention over configuration for
  managing environments and software deployment.
\item Show how convention over configuration can be used to simplify
  processes for managing changes from varying sources.
\item Illustrate how to manage project changes granularly or
  generally.
  \item Demonstrate usage of maven overlays and the kuali overlay
    maven archetype to automatically create an overlay for KFS, KC or Rice.
\end{itemize}

\section{Audience}
Technical track for
\begin{itemize}
\item Developers
\item Devops
\item Change managers
\end{itemize}

\subsection{Level}
Advanced

\section{Focus Area}
Technical

\section{Track}
Kuali Rice or Kuali Coeus or Kuali Financials

\section{Format}
Information Session

\end{document}
\documentclass[12pt,notitlepage]{article}
\author{Leo Przybylski}

\usepackage{fancyhdr}
\usepackage{graphicx}
\title{\includegraphics[width=0.75\textwidth]{kuali_base.png}\\CAS and
KIM LDAP Implementation for Developers}
\date{}
\pagestyle{fancy}
\fancyhead{} % Clear all header fields 
\fancyhead[OL]{\sectionmark}
\fancyhead[OR]{\includegraphics[height=26pt]{kuali_base.png}}% 
\fancyhead[ER]{\sectionmark}
\fancyhead[EL]{\includegraphics[height=26pt]{kuali_base.png}}% 


\begin{document}
\maketitle

\abstract{KIM's reference implementation requires a RDBMS datastore. For systems that work together, all will 
access to the same datastore for principal/entity information.
However, systems university wide will need
a common system for identity management and authentication. KIM must also tie-in 
authorization across software from the Kuali Foundation.

This session covers the ease of integrating Spring LDAP with KIM and the caveats of KIM integration and alternative
datastores. 

Attendees will see and experience taking an existing Kuali application and adding a directory service datastore for
entities. The goals are that attendees will learn about what goes into mapping a directory service to KIM, and how
to implement the necessary service overrides. They will also learn how
to setup CAS to communicate with KIM and how to manage developing with
LDAP integration and CAS implemented together.

\section{Presenters}
Leo Przybylski

\section{Audience}
Technical track for
\begin{itemize}
\item Developers
\item Devops
\end{itemize}


\subsection{Level}
Intermediate

\section{Focus Area}
Technical

\section{Track}
Kuali Financial System

\section{Format}
Information Session
\end{document}

\documentclass[xcolor=dvipsnames,14pt]{beamer}
\usetheme{kualidays2012}
\usecolortheme[RGB={125,25,25}]{structure}
\setbeamerfont{structure}{family=\rmfamily,series=\bfseries} 
\setbeamerfont{subtitle}{family=\sffamily,series=\bfseries} 
\setbeamercolor{normal text}{bg=brown!46}

\begin{document}

\title{Si Vis Bellum Para Bellum}
\subtitle{KC Customization using Maven's WAR overlays}
\author[Leo]{Leo Przybylski}

\institute[rSmart]{rSmart\inst{1} \\[1ex] 
  \texttt{leo@rsmart.com}
}

\begin{frame}[plain]
  \titlepage
\end{frame}

\begin{frame}{Overview}
  \begin{itemize}
  \item Welcome \& Collophon
  \item About Presenters
  \item Overlay Basics
  \item Overlay Patterns
  \end{itemize}
\end{frame}

\begin{frame}{Welcome and Collophon}
  \begin{itemize}
    \item Hi!
    \item ``Si Vis Bellum Para Bellum'' is a play on ``Si Vis Pacem
      Para Bellum'' and Web Archive (WAR) files. It means ``If you
      wish for war, prepare for war.'' War being a web archive for context.
  \end{itemize}
\end{frame}

\begin{frame}{About Presenters}
\end{frame}

\begin{frame}{Overlay Basics}
  \begin{itemize}
  \item First Overlay
  \item How Overlays work
  \end{itemize}
\end{frame}

\begin{frame}{First Overlay}
  Using KC as an Example to add CAS Support
  \begin{enumerate}
    \item Build KC.
    \item Install as a prototype.
    \item Setup prototype jar.
    \item Create overlay project.
    \item Add CAS dependency.
    \item Add dependencies for KC.
    \item Configure war plugin for overlay.
  \end{enumerate}
\end{frame}


\begin{frame}{How Overlays Work}
  \begin{itemize}
    \item A fileset or maven project copied over a webapp.
    \item Webapp does not have to be a maven project.
    \item The overlay does not have to be a complete webapp.
    \item An overlay does not even have to be very different at all.
  \end{itemize}
\end{frame}

\begin{frame}{What Overlays are Not}
  \begin{itemize}
  \item Project Inheritence.
  \item A replacement/alternative for multi-module projects.
  \item A silver-bullet for all webapp projects.
  \item Change management pattern.
  \end{itemize}
\end{frame}


\begin{frame}{Advantages of Overlays}
  \begin{itemize}
    \item Offers flexibility in your project by allowing granular
      modifications at the project level.
    \item Smaller projects for tracking changes with.
    \item Easier to modularize your project.
    \item Better code reuse.
  \end{itemize}
\end{frame}

\begin{frame}{Overlay Patterns}
  \begin{itemize}
    \item Overlaying configuration modifications.
    \item Overlay to activate a module.
    \item Overlay theming for UX/UI.
  \end{itemize}
\end{frame}

\begin{frame}{Overwriting Changes in an Overlaid Project}
%  \begin{itemize}
%  \end{itemize}
\end{frame}

\begin{frame}{Multi-module vs. Regular Webapp Project}
  \begin{itemize}
    \item Always difficult to build/manage multi-module webapp
      projects.
    \item Jar dependencies from the parent may not match up and
      therefore will not get overlaid (redundant jars).
  \end{itemize}
\end{frame}

\begin{frame}{Overlay Archetypes}
  \begin{itemize}
  \item Leo has developed an archetype for creating and overlay KC project.
  \end{itemize}
\end{frame}


\end{document}
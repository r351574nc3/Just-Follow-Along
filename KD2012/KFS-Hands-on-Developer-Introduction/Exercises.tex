\documentclass[letterpaper,notitlepage,12pt]{book}
\author{Leo Przybylski}
\usepackage[usenames,dvipsnames]{color}
\usepackage{minted}
\usepackage{graphicx}
\usepackage{epstopdf}
\usepackage{listings}
\usepackage{fancyhdr}
\usepackage{hyperref}
\hypersetup{
    colorlinks,
    citecolor=ForestGreen,
    filecolor=ForestGreen,
    linkcolor=ForestGreen,
    urlcolor=ForestGreen
}
\hypersetup{linktocpage}
%%\definecolor{ubergray}{RGB}{245,245,245}
\title{\includegraphics[width=\textwidth]{cover.png}\\KFS Hands-On Developer Introduction}
\date{}
\pagestyle{fancy}
\fancyhead{} % Clear all header fields 
\fancyhead[OL]{\sectionmark}
\fancyhead[OR]{\includegraphics[height=26pt]{heading.png}}% 
\fancyhead[ER]{\sectionmark}
\fancyhead[EL]{\includegraphics[height=26pt]{heading.png}}% 

\begin{document}
\maketitle
\tableofcontents

\addcontentsline{toc}{part}{Preface}
\part*{Preface}

\addcontentsline{toc}{section}{Copyright}
\section*{Copyright}
\addcontentsline{toc}{subsection}{Copyright Holder}
\subsection*{Copyright Holder}
\copyright Copyright 2011, 2012
Leo Przybylski
leo@rsmart.com

\addcontentsline{toc}{subsection}{Disclaimer}
\subsection*{Disclaimer}
This work is licensed under the Creative Commons Attribution-ShareAlike 3.0 United States License. To view a copy of this license, visit http://creativecommons.org/licenses/by-sa/3.0/us/ or send a letter to Creative Commons, 444 Castro Street, Suite 900, Mountain View, California, 94041, USA.

\addcontentsline{toc}{section}{About the Trainer}
\section*{About the Trainer}
Leo started working with the Kuali Foundation in 2005 as a developer
on the Kuali Financial System. Since then, he has worked as a
\emph{Development Manager} on the Kuali Financial System, \emph{Lead Developer} on
the Kuali Coeus project, \emph{Software Architect} on the University of
Arizona KFS implementation, and now is a \emph{Release Engineer} for the
Kuali Foundation for the Rice Project. 

Leo has given six presentations on KFS, KC, and Rice on to separate
Kuali Days occassions. 

One significant contribution he has made to the Kuali Community is his
Rice LDAP Integration module.

\addcontentsline{toc}{section}{Using these Exercises}
\section*{Using these Exercises}
\addcontentsline{toc}{subsection}{VirtualBox Appliance}
\subsection*{VirtualBox Appliance}
Exercise instructions are included in this document. All software and
examples are available on the VirtualBox appliance distributed during
class. To install the VirtualBox appliance:
\begin{enumerate}
  \item Execute the VirtualBox installer to install the software.
  \item Copy the \textbf{KFSDev.box} from the distributed USB drive to
    your hard disk.
  \item Execute from a shell 
    \begin{minted}{sh}
      vagrant box add KFSDev KFSDev.box
    \end{minted}
  \item Import the Virtual Machine
    \begin{minted}{sh}
      vagrant up
    \end{minted}
  \item Stop the Virtual Machine
    \begin{minted}{sh}
      vagrant halt
    \end{minted}
  \item Start VirtualBox
\end{enumerate}

\addcontentsline{toc}{subsection}{Virtual Machine Manifest}
\subsection*{Virtual Machine Manifest}
The VirtualBox appliance is an Ubuntu Linux distribution. Within it is
the software we will use for this class:
\begin{description}
  \item [Eclipse Indigo] the IDE used for class. Includes Subclipse,
    the m2eclipse plugin, and pre-installed projects with examples.
  \item [OpenJDK 1.7.0\_06 IcedTea] the JVM used for executing/testing
    examples.
  \item [Maven 3] used to build Rice applications, run tests, and
    start the Tomcat6 application
\end{description}

\subsubsection*{Credentials}
\begin{description}
\item [User Account] is \textbf{kuali} with the password
  \textbf{kuali}. This is used to unlock the VM after it has suspended,
  gone to sleep, or locked. The password is also required for
  executing commands as \textbf{root} which may on occassion be
  required. The user account home directory is located at
  \textbf{/home/kuali} and will frequently be referred to during the training.
\item [Database Account] uses the jdbc connection string
  \textbf{jdbc:mysql://localhost:3306/kuldemo} and the
  username/password \textbf{kuldemo}/\textbf{kuldemo}. These are the
  default credentials and database connection information as defined
  in kul-cfg-dbs.
\end{description}

\subsubsection*{Structure}
The Eclipse workspace is located at \textbf{/home/kuali/workspace}. 


\addcontentsline{toc}{section}{Training Overview}
\section*{Training Overview}

\addcontentsline{toc}{part}{Exercise 1: Import Project}
\part*{Exercise 1:\\
Import Project}

\addcontentsline{toc}{section}{Exercise 1: Import Project}
{\setlength{\baselineskip}%
  {0.0\baselineskip}
  \section*{\flushright Exercise 1\\
Import Project}
  \hrulefill \par}

\addcontentsline{toc}{subsection}{Description}
\subsection*{Description}

\addcontentsline{toc}{subsection}{Goals}
\subsection*{Goals}

\addcontentsline{toc}{subsection}{Instructions}
\subsection*{Instructions}


\newpage
{\setlength{\baselineskip}%
  {0.0\baselineskip}
  \section*{Notes}
  \hrulefill \par}
\addcontentsline{toc}{part}{Exercise 2: Database Setup}
\part*{Exercise 2: \\
Database Setup}

\addcontentsline{toc}{section}{Exercise 2: Database Setup}
{\setlength{\baselineskip}%
  {0.0\baselineskip}
  \section*{\flushright Exercise 2: \\
Database Setup}
  \hrulefill \par}

\addcontentsline{toc}{subsection}{Description}
\subsection*{Description}

\addcontentsline{toc}{subsection}{Goals}
\subsection*{Goals}

\addcontentsline{toc}{subsection}{Instructions}
\subsection*{Instructions}

\newpage
{\setlength{\baselineskip}%
  {0.0\baselineskip}
  \section*{Notes}
  \hrulefill \par}
\addcontentsline{toc}{part}{Exercise 3: Create Business Object Table}
\part*{Exercise 3: \\
Create Business Object Table}

\addcontentsline{toc}{section}{Exercise 3: Create Business Object Table}
{\setlength{\baselineskip}%
  {0.0\baselineskip}
  \section*{\flushright Exercise 3: \\
Create Business Object Table}
  \hrulefill \par}

\addcontentsline{toc}{subsection}{Description}
\subsection*{Description}

\addcontentsline{toc}{subsection}{Goals}
\subsection*{Goals}

\addcontentsline{toc}{subsection}{Instructions}
\subsection*{Instructions}
\subsubsection*{1 Run mysql}
\begin{minted}{sh}
  mysql -u kuldev -p kuldev
\end{minted}
\newpage
{\setlength{\baselineskip}%
  {0.0\baselineskip}
  \section*{Notes}
  \hrulefill \par}
\addcontentsline{toc}{part}{Exercise 4: Create Maintenance Document}
\part*{Exercise 4: \\
Setup Business Object}

\addcontentsline{toc}{section}{Exercise 5: Create Maintenance Document}
{\setlength{\baselineskip}%
  {0.0\baselineskip}
  \section*{\flushright Exercise r: \\
Setup Business Object}
  \hrulefill \par}

\addcontentsline{toc}{subsection}{Description}
\subsection*{Description}

\addcontentsline{toc}{subsection}{Goals}
\subsection*{Goals}

\addcontentsline{toc}{subsection}{Instructions}
\subsection*{Instructions}

\subsubsection*{1 Create Business Object Class}
Create a file in \textbf{work/src/org/kuali/kfs/sys/businessobject/}
called \textbf{FerpaCertification}
\begin{minted}{java}
/*
 * Copyright 2005-2008 The Kuali Foundation
 * 
 * Licensed under the Educational Community License, Version 2.0 (the "License");
 * you may not use this file except in compliance with the License.
 * You may obtain a copy of the License at
 * 
 * http://www.opensource.org/licenses/ecl2.php
 * 
 * Unless required by applicable law or agreed to in writing, software
 * distributed under the License is distributed on an "AS IS" BASIS,
 * WITHOUT WARRANTIES OR CONDITIONS OF ANY KIND, either express or implied.
 * See the License for the specific language governing permissions and
 * limitations under the License.
 */
package org.kuali.kfs.sys.businessobject;


/**
 * Ferpa Certification Business Object
 */
public class FerpaCertification extends PersistableBusinessObjectBase implements Inactivateable {
  
  private Long id;
  private String principalId;
  private String Person;

  /**
  * Default no-arg constructor.
  */
  
  public FerpaCertification() {
        super();
    }

    /**
     * Gets the id attribute.
     * 
     * @return Returns the id.
     */
    public String getId() {
        return id;
    }


    /**
     * Sets the id attribute value.
     * 
     * @param id The id to set.
     */
    public void setId(Long id) {
        this.id = id;
    }


    /**
     * Gets the principalId attribute.
     * 
     * @return Returns the principalId.
     */
    public Long getPrincipalId() {
        return principalId;
    }


    /**
     * Sets the principalId attribute value.
     * 
     * @param principalId The principalId to set.
     */
    public void setPrincipalId(String principalId) {
        this.principalId = principalId;
    }


    /**
     * Gets the person attribute.
     * 
     * @return Returns the person.
     */
    public String getPerson() {
        return person;
    }


    /**
     * Sets the person attribute value.
     * 
     * @param person The person to set.
     */
    public void setPerson(String person) {
        this.person = person;
    }
}
\end{minted}

\subsubsection*{2 Create Business DataDictionary Metadata}
\begin{minted}{xml}
<?xml version="1.0" encoding="UTF-8"?><beans xmlns="http://www.springframework.org/schema/beans" xmlns:xsi="http://www.w3.org/2001/XMLSchema-instance" xmlns:p="http://www.springframework.org/schema/p" xsi:schemaLocation="http://www.springframework.org/schema/beans         http://www.springframework.org/schema/beans/spring-beans-2.0.xsd">
<!--
 Copyright 2008-2009 The Kuali Foundation
 
 Licensed under the Educational Community License, Version 2.0 (the "License");
 you may not use this file except in compliance with the License.
 You may obtain a copy of the License at
 
 http://www.opensource.org/licenses/ecl2.php
 
 Unless required by applicable law or agreed to in writing, software
 distributed under the License is distributed on an "AS IS" BASIS,
 WITHOUT WARRANTIES OR CONDITIONS OF ANY KIND, either express or implied.
 See the License for the specific language governing permissions and
 limitations under the License.
-->

  <bean id="FerpaCertification" parent="FerpaCertification-parentBean"/>

  <bean id="FerpaCertification-parentBean" abstract="true" parent="BusinessObjectEntry">
    <property name="businessObjectClass" value="org.kuali.kfs.sys.businessobject.FerpaCertification"/>
    <property name="inquiryDefinition">
      <ref bean="FerpaCertification-inquiryDefinition"/>
    </property>
    <property name="lookupDefinition">
      <ref bean="FerpaCertification-lookupDefinition"/>
    </property>
    <property name="titleAttribute" value="bankCode"/>
    <property name="objectLabel" value="FerpaCertification"/>
    <property name="attributes">
      <list>
        <ref bean="FerpaCertification-bankCode"/>
        <ref bean="FerpaCertification-bankName"/>
        <ref bean="FerpaCertification-bankShortName"/>
        <ref bean="FerpaCertification-bankRoutingNumber"/>
        <ref bean="FerpaCertification-bankAccountNumber"/>
        <ref bean="FerpaCertification-bankAccountDescription"/>
        <ref bean="FerpaCertification-cashOffsetFinancialChartOfAccountCode"/>
        <ref bean="FerpaCertification-cashOffsetAccountNumber"/>
        <ref bean="FerpaCertification-cashOffsetSubAccountNumber"/>
        <ref bean="FerpaCertification-cashOffsetObjectCode"/>
        <ref bean="FerpaCertification-cashOffsetSubObjectCode"/>
        <ref bean="FerpaCertification-continuationFerpaCertificationCode"/>
        <ref bean="FerpaCertification-bankDepositIndicator"/>
        <ref bean="FerpaCertification-bankDisbursementIndicator"/>
        <ref bean="FerpaCertification-bankAchIndicator"/>
        <ref bean="FerpaCertification-bankCheckIndicator"/>
        <ref bean="FerpaCertification-active"/>
      </list>
    </property>
  </bean>

<!-- Attribute Definitions -->


  <bean id="FerpaCertification-id" parent="FerpaCertification-id-parentBean"/>

  <bean id="FerpaCertification-id-parentBean" abstract="true" parent="AttributeDefinition">
    <property name="name" value="id"/>
    <property name="forceUppercase" value="true"/>
    <property name="label" value="FerpaCertification Code"/>
    <property name="shortLabel" value="FerpaCertification Code"/>
    <property name="maxLength" value="4"/>
    <property name="validationPattern">
      <ref bean="AlphaNumericValidation" />
    </property>
    <property name="required" value="true"/>
    <property name="control">
      <ref bean="HiddenControl" />
    </property>
  </bean>
  <bean id="FerpaCertification-principalId" parent="FerpaCertification-principalId-parentBean"/>

  <bean id="FerpaCertification-principalId-parentBean" abstract="true" parent="PersonImpl-principalId">
    <property name="name" value="principalId"/>
  </bean>

  <bean id="FerpaCertification-person.principalName" parent="FerpaCertification-person.principalName-parentBean"/>

  <bean id="FerpaCertification-person.principalName-parentBean" abstract="true" parent="PersonImpl-principalName">
    <property name="name" value="person.principalName"/>
    <property name="control">
      <bean parent="KualiUserControlDefinition" p:personNameAttributeName="person.name" p:universalIdAttributeName="principalId" p:userIdAttributeName="person.principalName"/>
    </property>
    <property name="required" value="false"/>
  </bean>

  <bean id="FerpaCertification-active" parent="FerpaCertification-active-parentBean"/>      
  <bean id="FerpaCertification-active-parentBean" abstract="true" parent="GenericAttributes-activeIndicator">
    <property name="name" value="active"/>
    <property name="required" value="false"/>
  </bean>
  
<!-- Business Object Inquiry Definition -->


  <bean id="FerpaCertification-inquiryDefinition" parent="FerpaCertification-inquiryDefinition-parentBean"/>

  <bean id="FerpaCertification-inquiryDefinition-parentBean" abstract="true" parent="InquiryDefinition">
    <property name="title" value="FerpaCertification Inquiry"/>
    <property name="inquirySections">
      <list>
        <bean parent="InquirySectionDefinition">
          <property name="title" value=""/>
          <property name="numberOfColumns" value="1"/>
          <property name="inquiryFields">
            <list>
              <bean parent="FieldDefinition" p:attributeName="id"/>
              <bean parent="FieldDefinition" p:attributeName="principalId"/>
              <bean parent="FieldDefinition" p:attributeName="active"/>
            </list>
          </property>
        </bean>
      </list>
    </property>
  </bean>
  
<!-- Business Object Lookup Definition -->


  <bean id="FerpaCertification-lookupDefinition" parent="FerpaCertification-lookupDefinition-parentBean"/>

  <bean id="FerpaCertification-lookupDefinition-parentBean" abstract="true" parent="LookupDefinition">
    <property name="title" value="FerpaCertification Lookup"/>
    
    <property name="defaultSort">
      <bean parent="SortDefinition">
        <property name="attributeNames">
          <list>
            <value>bankCode</value>
          </list>
        </property>
      </bean>
    </property>
    <property name="lookupFields">
      <list>
              <bean parent="FieldDefinition" p:attributeName="principalId"/>
              <bean parent="FieldDefinition" p:attributeName="active"/>
      </list>
    </property>
    <property name="resultFields">
      <list>
              <bean parent="FieldDefinition" p:attributeName="person.principalName"/>
              <bean parent="FieldDefinition" p:attributeName="active"/>
      </list>
    </property>
  </bean>
</beans>\end{minted}

\subsubsection*{3 Create O/R Mapping}

\begin{minted}{xml}
\end{minted}

\subsubsection*{Add a link to the portal}

\newpage
{\setlength{\baselineskip}%
  {0.0\baselineskip}
  \section*{Notes}
  \hrulefill \par}
\addcontentsline{toc}{part}{Exercise 5: Create Maintenance Document}
\part*{Exercise 5: \\
Create Maintenance Document}

\addcontentsline{toc}{section}{Exercise 5: Create Maintenance Document}
{\setlength{\baselineskip}%
  {0.0\baselineskip}
  \section*{\flushright Exercise 5: \\
Create Maintenance Document}
  \hrulefill \par}

\addcontentsline{toc}{subsection}{Description}
\subsection*{Description}

\addcontentsline{toc}{subsection}{Goals}
\subsection*{Goals}

\addcontentsline{toc}{subsection}{Instructions}
\subsection*{Instructions}

\newpage
{\setlength{\baselineskip}%
  {0.0\baselineskip}
  \section*{Notes}
  \hrulefill \par}
\addcontentsline{toc}{part}{Exercise 6: Basic Routing}
\part*{Exercise 6: \\
Basic Routing}

\addcontentsline{toc}{section}{Exercise 6: Basic Routing}
{\setlength{\baselineskip}%
  {0.0\baselineskip}
  \section*{\flushright Exercise 6: \\
Basic Routing}
  \hrulefill \par}

\addcontentsline{toc}{subsection}{Description}
\subsection*{Description}

\addcontentsline{toc}{subsection}{Goals}
\subsection*{Goals}

\addcontentsline{toc}{subsection}{Instructions}
\subsection*{Instructions}

\newpage
{\setlength{\baselineskip}%
  {0.0\baselineskip}
  \section*{Notes}
  \hrulefill \par}
\addcontentsline{toc}{part}{Exercise 7: Split Node}
\part*{Exercise 7: \\
Split Node}

\addcontentsline{toc}{section}{Exercise 7: Split Node}
{\setlength{\baselineskip}%
  {0.0\baselineskip}
  \section*{\flushright Exercise 7: \\
Split Node}
  \hrulefill \par}

\addcontentsline{toc}{subsection}{Description}
\subsection*{Description}

\addcontentsline{toc}{subsection}{Goals}
\subsection*{Goals}

\addcontentsline{toc}{subsection}{Instructions}
\subsection*{Instructions}

\subsubsection*{1 Create FerpaCertificationMaintainable.java}
\begin{minted}{java}
/*
 * Copyright 2009 The Kuali Foundation
 * 
 * Licensed under the Educational Community License, Version 2.0 (the "License");
 * you may not use this file except in compliance with the License.
 * You may obtain a copy of the License at
 * 
 * http://www.opensource.org/licenses/ecl2.php
 * 
 * Unless required by applicable law or agreed to in writing, software
 * distributed under the License is distributed on an "AS IS" BASIS,
 * WITHOUT WARRANTIES OR CONDITIONS OF ANY KIND, either express or implied.
 * See the License for the specific language governing permissions and
 * limitations under the License.
 */
package org.kuali.kfs.module.ar.document;

import org.apache.commons.lang.StringUtils;
import org.kuali.kfs.module.ar.businessobject.InvoiceRecurrence;
import org.kuali.kfs.sys.KFSConstants;
import org.kuali.kfs.sys.context.SpringContext;
import org.kuali.kfs.sys.document.FinancialSystemMaintainable;
import org.kuali.kfs.sys.document.FinancialSystemMaintenanceDocument;
import org.kuali.rice.kew.exception.WorkflowException;
import org.kuali.rice.kim.bo.Person;
import org.kuali.rice.kim.bo.entity.KimPrincipal;
import org.kuali.rice.kim.service.IdentityManagementService;
import org.kuali.rice.kns.service.DocumentService;
import org.kuali.rice.kns.workflow.service.KualiWorkflowDocument;

public class InvoiceRecurrenceMaintainable extends FinancialSystemMaintainable {

\end{minted}

\subsubsection*{2 Implement answerSplitNodeQuestion in FerpaCertificationMaintainable}

\begin{minted}{java}
@Override
protected boolean answerSplitNodeQuestion(String nodeName) throws UnsupportedOperationException {
}    
\end{minted}

\newpage
{\setlength{\baselineskip}%
  {0.0\baselineskip}
  \section*{Notes}
  \hrulefill \par}

\end{document}

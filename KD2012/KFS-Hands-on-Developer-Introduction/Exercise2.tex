\addcontentsline{toc}{part}{Exercise 2: Database Setup}
\part*{Exercise 2: \\
Database Setup}

\addcontentsline{toc}{section}{Exercise 2: Database Setup}
{\setlength{\baselineskip}%
  {0.0\baselineskip}
  \section*{\flushright Exercise 2: \\
Database Setup}
  \hrulefill \par}

\addcontentsline{toc}{subsection}{Description}
\subsection*{Description}

\addcontentsline{toc}{subsection}{Goals}
\subsection*{Goals}

\addcontentsline{toc}{subsection}{Instructions}
\subsection*{Instructions}
\subsubsection{1 Update impex-build.properties}
Open \textbf{/home/kuali/impex-build.properties}. Locate the section of code that looks like
\begin{minted}{yaml}
import.torque.database = mysql
import.torque.database.driver = com.mysql.jdbc.Driver
import.torque.database.url = jdbc:mysql://localhost:3306/kuldemo
import.torque.database.user=kuldemo
import.torque.database.schema=KULDEMO
import.torque.database.password=kuldemo
\end{minted}

and change it to

\begin{minted}{yaml}
import.torque.database = mysql
import.torque.database.driver = com.mysql.jdbc.Driver
import.torque.database.url = jdbc:mysql://localhost:3306/kuldev
import.torque.database.user=kuldev
import.torque.database.schema=KULdev
import.torque.database.password=kuldev
\end{minted}

\subsubsection{1 Update kfs-build.properties}
Open \textbf{/home/kuali/kfs-build.properties} Locate the section of code that looks like
\begin{minted}{yaml}
datasource.username=kuldemo
datasource.password=kuldemo
mysql.datasource.url=jdbc:mysql://localhost:3306/kuldemo
\end{minted}

and change it to

\begin{minted}{yaml}
datasource.username=kuldev
datasource.password=kuldev
mysql.datasource.url=jdbc:mysql://localhost:3306/kuldev
\end{minted}

\subsubsection{2 Run Impex}
Using a terminal window do the following:
\begin{minted}{sh}
  cd /home/kuali/workspace/kul-cfg-dbs/impex
  ant create-schema import
\end{minted}

\newpage
{\setlength{\baselineskip}%
  {0.0\baselineskip}
  \section*{Notes}
  \hrulefill \par}
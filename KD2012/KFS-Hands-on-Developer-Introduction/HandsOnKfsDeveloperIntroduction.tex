\documentclass[xcolor=dvipsnames,14pt]{beamer}
\usepackage{minted}
\usetheme{kualidays2012}
\usecolortheme[RGB={125,25,25}]{structure}
\setbeamerfont{structure}{family=\rmfamily,series=\bfseries} 
\setbeamerfont{subtitle}{family=\sffamily,series=\bfseries} 
\setbeamercolor{normal text}{bg=brown!46}

\begin{document}

\title[A short proof]{Getting Your Hands Dirty with KFS}
\subtitle[Errors]{KFS Hands On Developer Introduction}
\author[Leo]{Leo Przybylski}

\institute[rSmart]{rSmart\inst{1} \\[1ex] 
  \texttt{leo@rsmart.com}
}


\begin{frame}[plain]
  \titlepage
\end{frame}

\begin{frame}{Instructors}
	\begin{itemize}
		\item Jamey Decker
		\item Leo Przybylski
	\end{itemize}
\end{frame}

\begin{frame}{Overview}
	\begin{itemize}
		\item About the Class
		\begin{itemize}
			\item Virtual Machines
		\end{itemize}
        \item Exercises
	\end{itemize}
\end{frame}

\begin{frame}[fragile]{Exercise 1: Import Project}
	\begin{itemize}
		\item Goals
          \begin{itemize}
            \item Introduction to Eclipse and KFS
            \item Understand importing KFS into Eclipse from source
          \end{itemize}          
		\item Tasks
          \begin{itemize}
            \item Import KFS into Eclipse
          \end{itemize}          
	\end{itemize}

\end{frame}

\begin{frame}{Exercise 2: Database Setup}
	\begin{itemize}
		\item Goals
          \begin{itemize}
            \item Familiarize with the impex tool
            \item Understand how to configure the impex tool
            \item Learn how the exported database is structured and loaded
          \end{itemize}          
		\item Tasks
          \begin{itemize}
          \item Locate impex within the workspace
          \item Modify the impex-build.properties to create a new
            database
          \item Begin the database import process
          \end{itemize}          

	\end{itemize}
\end{frame}

\begin{frame}{FERPA Certification}
  \begin{itemize}
    \item Used to process what people have been FERPA certified
    \item Allows us to use verify FERPA certification from other documents
    \item Not a real FERPA implementation. Just an example on how
      MaintenanceDocuments and Business Objects in KFS can be used.
    \item Setting up routing based on FERPA Certification
  \end{itemize}
\end{frame}

\begin{frame}{Exercise 3: Create Business Object Table}
	\begin{itemize}
		\item Goals
          \begin{itemize}
            \item Understand how the OR/M, DataDictionary, and Java
              classes map class information and metadata for Business
              Objects and Documents
            \item See common patterns for creating tables in KFS
            \item Understand naming conventions
          \end{itemize}          
		\item Tasks
          \begin{itemize}
            \item Create a database table with constraints
          \end{itemize}          
	\end{itemize}
\end{frame}

\begin{frame}{Exercise 4: Setup Business Object}
	\begin{itemize}
		\item Goals
          \begin{itemize}
            \item Understand how the OR/M, DataDictionary, and Java
              classes map class information and metadata for Business
              Objects and Documents
            \item Learn patterns for creating Business Objects
            \item Understand controls, lookups, and inquiries and how
              to configure them
          \end{itemize}          
		\item Tasks	
          \begin{itemize}
            \item Create a FerpaCertification class
            \item Create a DataDictionary BusinessObjectEntry
          \end{itemize}          
\end{itemize}
\end{frame}

\begin{frame}{Exercise 4: Create Maintenance Document}
	\begin{itemize}
		\item Goals
          \begin{itemize}
            \item Understand how the OR/M, DataDictionary, and Java
              classes map class information and metadata for Business
              Objects and Documents
            \item Learn how to add fields to a document and modify controls.
          \end{itemize}          
		\item Tasks	
          \begin{itemize}
            \item Create a DataDictionary MaintenanceDocumentEntry
          \end{itemize}          
        \end{itemize}
\end{frame}

\begin{frame}{DataDictionary Demystified: Business Objects and
    Maintenance Documents}
  \begin{itemize}
    \item MaintenanceDocuments are simplifying the user interface
      design of a document/page.
    \item BusinessObject entries are for describing metadata about the
      business object for use in other documents/objects. Some of the
      information may not be necessary for a document.
    \item BusinessObject entries also define metadata about lookups
      and inquiries.
  \end{itemize}           
\end{frame}

\begin{frame}{Exercise 5: Basic Routing}
	\begin{itemize}
		\item Goals
          \begin{itemize}
            \item Understand how to create and then ingest workflow doctypes
          \end{itemize}          
		\item Tasks
          \begin{itemize}
            \item Create a workflow doctype for FerpaCertificationDocument
            \item Ingest the workflow doctype into Rice using the ingester
          \end{itemize}          
	\end{itemize}
\end{frame}

\begin{frame}{Exercise 6: Split Node}
	\begin{itemize}
		\item Goals
          \begin{itemize}
            \item Learn about how to create more complicated workflows
              based on decisions and document information
            \item Understand what java and xml needs to be modified
              for split nodes
          \end{itemize}          
		\item Tasks
          \begin{itemize}
            \item Add split node routing information to the doctype
            \item Add a method to handle split nodes to the document
          \end{itemize}          
	\end{itemize}
\end{frame}

\end{document}
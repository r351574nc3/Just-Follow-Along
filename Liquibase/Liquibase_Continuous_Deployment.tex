\documentclass[12pt,notitlepage]{article}
\author{Leo Przybylski}
\usepackage{fancyhdr}
\usepackage{graphicx}
\usepackage{hyperref}
\usepackage{listings}
\usepackage[usenames,dvipsnames]{color}

\title{\includegraphics[width=0.75\textwidth]{../images/rsmart_base.png}\\\includegraphics[width=0.70\textwidth]{../images/kuali_base.png}\\rSmart
Continuous Deployment with Liquibase}
\pagestyle{fancy}
\fancyhead{} % Clear all header fields 
\fancyhead[OL]{\sectionmark}
\fancyhead[OR]{\includegraphics[height=26pt]{../images/rsmart_base.png}}% 
\fancyhead[ER]{\sectionmark}
\fancyhead[EL]{\includegraphics[height=26pt]{../images/rsmart_base.png}}% 
\definecolor{ubergray}{RGB}{245,245,245}
\hypersetup{colorlinks}
\lstset{basicstyle=\scriptsize,
  backgroundcolor=\color{ubergray},
  breaklines=true,
  frame=single,
  includerangemarker=false}
\begin{document}
\maketitle
\tableofcontents

\abstract{}

\section{Goals}

The development process has some specific goals. Really, we had some
specific questions we wanted answered and we want to only have to
answer them once. We don't want to have to keep asking these
questions. DRY (Do not repeat yourself). If we can answer these questions and find a way that we
will not have to answer them anymore, then we have simplified our
development process a great deal.

\subsection{Development}

One of the areas we want to simplify is development. Kuali has
standardized on Maven who's principal tenet is ``Convention over
Configuration.'' Basically, by defining standards and common practices
across domains, we can minimize overhead. Let's analyze areas on our
current KFS development projects where we have separation instead of
standards within the domain.

\subsubsection{Commits}
The standard is to use the Jira issue name with a commit, so the
change can be tracked from Jira. There are two problems we currently
have here. With new developers, we always have to emphasize the
importance of this. This is one area we always repeat ourselves. Also,
sometimes a jira issue is not created for a certain task. Rather than
create an issue, the developer completes the task.

Another problem is with multiple checkins. Sometimes with a change, a
developer does not do it in one checkin and things get missed. A
developer then has to go back and grab the other files that are
missing. We keep emphasizing the importance of complete checkins, but
human error always interferes. How do we keep from repeating ourselves
here?

\subsubsection{Configuration}Not to be confused with Configuration
Management, this is rather project configuration
per-developer. In KFS, we rely a lot on project configuration being
similar. Unfortunately, with multiple databases, project-specific
dependencies, etc..., it is difficult to maintain a common
configuration. Not to mention, if developers are working on more than
one project at a time, how do we stay on top of project
configurations?

Another issue with configurations is that developers do not follow the
same standards. They tend to follow the way they are taught, but not
all developers have the same teacher. This leads to project/directory
structures and even deveopment styles do differ. Having inconsistent
configurations can cause confusion and maintenance hassles. How do we
stop repeating ourselves here?

\subsubsection{Building}
This is partially to do with configuration, but the biggest problem is
the length at which builds take, the numerous steps to produce a clean
build, and the need to build often. This has been a problem since the
beginning of Kuali. Many have solved this problem, but not many are
practicing the solutions. We need to start following these practices
at rSmart to mitigate the time we spend hassling with builds.

\subsubsection{Testing}
We boast a lot about TDD, but we have not been practicing what we
preach. Is it a matter of discipline? Or habit? How do we keep asking
ourselves this and solve it once and for all? Many times, institutions
ask us for our input/contribution on automated testing, but we
constantly have to build from scratch because we have nothing to
begin with.

\subsection{Configuration Management}
\begin{itemize}
  \item Consistent configurations for all platforms
  \item Manage configurations in VCS
    \begin{itemize}
      \item allows common config to be maintained by revision
      \end{itemize}
    \item Frequent releases with changes coupled to a Jira version
    \item Better process management tool integration
    \item No work without Jira issue
\end{itemize}


\section{Use Cases}

Some use cases I put together to illustrate ideal solutions that
answer some of the questions above. Ideally, we want to answer them
all, but these are based on assumptions I can think of.

\subsection{Project Creation}

Creating a new KFS project for example:

\begin{lstlisting}[language=bash,caption={Create a new project with maven}]
$ mvn rsmart-kfs:create-overlay -DinstitutionId=edu.stevens
-DjiraProject=STEVENS -DprototypeId=com.rsmart.kuali.kfs:kfs:5.0
\end{lstlisting}

The above requires
\begin{itemize}
\item Jira project named stevens created. It contains information in
  its description in YAML format which the mvn tool will use to
  generate a new project using the rsmart-kfs archetype.
\item Since it is creating an overlay, this implies that in rsmart's
  maven repository, there exists a prototype with the GAV
  com.rsmart.kuali.kfs:kfs:5.0
\item We are defining this for the stevens institution, so it will
  create a new project with the groupId edu.stevens.kuali.kfs and
  pullover the necessary configuration information from the prototype
\end{itemize}

\subsection{New Prototype}

Creating a new prototype. A prototype is what is used to create
overlays on top of. This is usually base code. I expect most of the
prototypes to be either from the org.kuali.kfs group or the
com.rsmart.kuali.kfs group.

\begin{lstlisting}[language=bash,caption={Deploy a new Prototype}]
$ mvn rsmart-kfs:create-prototype -DgroupId=com.rsmart -DprototypeId=com.rsmart.kuali.kfs:kfs:5.0
\end{lstlisting}

A prototype doesn't need a jira project. In fact, it probably doesn't
even get checked into SVN. It is simply a war that is built and
deployed to the maven repository specifically for overlays to be built
on top of it.

\subsection{Environment Setup}

From the path where you want your environment created:

\begin{lstlisting}[language=bash,caption={Create a new development environment with maven}]
$ mvn rsmart-kfs:setup-development -DjiraId=STEVENS -DdeveloperId=lprzybylski
\end{lstlisting}

This will go and grab the necessary information from the jira project
and build a local development for the lprzybylski user. This includes
checking out the project and setting up configuration. The
configuration is exclusive from other environments/projects. For
example, stevens does not use config or step on config from
cornell. The project information is encapsulated from the path where
the above is run; therefore, deleting the path or copying the path
will copy/remove all project related data.

Notice the developerId is captured and stored in local project
metadata. This is important for simplying tasks mentioned later. The
jira id is also added to local project metadata. The above settings
can be overridden, but are now used as defaults.

\subsection{Start Work}

\begin{lstlisting}[language=bash,caption={Assign the above issue to me}]
$ mvn rsmart-kfs:assign -DissueId=STEVENS-12
\end{lstlisting}

Assigns the given issue. This is not meant to replace the Jira UI at
all. Rather, it is intended to use jira more ubiquitously. The problem
we are attempting to solve is that there's 3 tools for change
management we are trying to merge. Jira, SVN, and Maven. Instead of
referring, to SVN revisions and tracking things by SVN revision, we
can now track things by Jira issue id.

By executing the above, the following will happen:
\begin{enumerate}
  \item Lookup Jira Issue
  \item Check Jira issue for attached changelist.txt
  \item Create changelist.txt if necessary
  \item Assign the issue to self
\end{enumerate}

Notice it uses something called a \emph{changelist.txt}. I want to
draw attention to this because a changelist is going to become
instrumental in managing changes acrossed jira issues. It is explained
better in
\href{http://svnbook.red-bean.com/en/1.7/svn.advanced.changelists.html}{The
  Red Bean Book}. Basically, there will be a new path called
changelists/ attached to the project. A changelist will be prefixed
with the jira issue. For example, STEVENS-12-changelist.txt. This
changelist will contain a list of all the files modified for this Jira
issue. It will be helpful because of switching between Jira issues,
what is lost is what files were modified. True, a developer could look
at the svn commits, but it would be much easier to contain these
within the Jira issue. Whenever a jira issue is assigned, the
developer can attain the changelist immediately within eclipse. See
\href{http://subclipse.tigris.org/issues/show_bug.cgi?id=635}{Subclipse
changelist support}. Basically, if a changelist exists for an issue, a
developer can get ahold of all the files modified in that jira issue
through the changelist.

Changelists are also helpful with cherrypicking (see later).

\newpage

\subsection{Commit Change}

A user can commit a blanket change like this:

\begin{lstlisting}[language=bash,caption={Commit a change to an issue}]
$ mvn rsmart-kfs:commit -DissueId=STEVENS-12
Give a Message:
 __     __   ______   ______                                      
/\ \  _ \ \ /\  __ \ /\  __ \                                     
\ \ \/ ".\ \\ \ \/\ \\ \  __ \                                    
 \ \__/".~\_\\ \_____\\ \_\ \_\                                   
  \/_/   \/_/ \/_____/ \/_/\/_/                                   
                                                                  
 ______  __  __   __   ______                                     
/\__  _\/\ \_\ \ /\ \ /\  ___\                                    
\/_/\ \/\ \  __ \\ \ \\ \___  \                                   
   \ \_\ \ \_\ \_\\ \_\\/\_____\                                  
    \/_/  \/_/\/_/ \/_/ \/_____/                                  
                                                                  
 ______   __   __   ______       __     __   ______   ______      
/\  __ \ /\ "-.\ \ /\  ___\     /\ \  _ \ \ /\  __ \ /\  ___\     
\ \ \/\ \\ \ \-.  \\ \  __\     \ \ \/ ".\ \\ \  __ \\ \___  \    
 \ \_____\\ \_\\"\_\\ \_____\    \ \__/".~\_\\ \_\ \_\\/\_____\   
  \/_____/ \/_/ \/_/ \/_____/     \/_/   \/_/ \/_/\/_/ \/_____/   
                                                                  
 ______   ______   ______   __       __       __  __              
/\  == \ /\  ___\ /\  __ \ /\ \     /\ \     /\ \_\ \             
\ \  __< \ \  __\ \ \  __ \\ \ \____\ \ \____\ \____ \            
 \ \_\ \_\\ \_____\\ \_\ \_\\ \_____\\ \_____\\/\_____\           
  \/_/ /_/ \/_____/ \/_/\/_/ \/_____/ \/_____/ \/_____/           
                                                                  
 __  __   ______   ______   _____                                 
/\ \_\ \ /\  __ \ /\  == \ /\  __-.                               
\ \  __ \\ \  __ \\ \  __< \ \ \/\ \                              
 \ \_\ \_\\ \_\ \_\\ \_\ \_\\ \____-                              
  \/_/\/_/ \/_/\/_/ \/_/ /_/ \/____/                              
\end{lstlisting}

Or itemize change per file:

\begin{lstlisting}[language=bash,caption={Commit a change to an issue}]
$ mvn rsmart-kfs:commit -DissueId=STEVENS-12 -Ditemize=true
Do you want to add the following files to STEVENS-12-changelist.txt?
M             2618
work/src/org/kuali/kfs/sys/context/PropertyLoadingFactoryBean.java

y(N): y

Do you want to add the following files to the project?
?                    build/war
?                    KFSMI-8503.diff

y(N): n

Give a Message:

* file1:
Creating new configuration for blah

* file.xml
Adding database structure

* file2.xml 
Creating a new parameter for blah

\end{lstlisting}

\subsection{Two or More Issues}

One common practice that is being used for change management is the
concept of cherry picking
(\href{http://svnbook.red-bean.com/en/1.7/svn.branchmerge.advanced.html}
and
\href{http://www.kernel.org/pub/software/scm/git/docs/git-cherry-pick.html}). This
is used to be able to add/remove changes from a release in an agile
way. We are going to need this very much. One best practice to help
facilitate cherry picking is one issue at a time with very few
commits. The trouble with this practice is that there is always an
occassion when an issue becomes blocked and a developer cannot just
sit and wait. It is inevitable there will be a need to work on 2
issues at once. Working on 2 issues at the same time that modify the
same file make cherry picking virtually impossible. How can we resolve
this?

When using the following command (also mentioned above)
\begin{lstlisting}[language=bash,caption={Assign the above issue to me}]
$ mvn rsmart-kfs:assign -DissueId=STEVENS-12
You are currently working on STEVENS-15. You also have uncommitted
changes. Before working on STEVENS-12, would you like to stash your
work for later? y(N): y
Creating
https://svn.rsmart.com/svn/kuali/kfs/rsmart_kfs_stevens/stashes/leo/STEVENS-12...done.
Checking out https://svn.rsmart.com/svn/kuali/kfs/rsmart_kfs_stevens/stashes/leo/STEVENS-12...done.
Committing to https://svn.rsmart.com/svn/kuali/kfs/rsmart_kfs_stevens/stashes/leo/STEVENS-12...done.
Switching to
https://svn.rsmart.com/svn/kuali/kfs/rsmart_kfs_stevens/trunk...done.

You are now on trunk at revision 2618. STEVENS-12 is still assigned to
you, but no longer "In Progress". To switch back use:
mvn rsmart-kfs:switch -DissueId=STEVENS-12
\end{lstlisting}

Maven will check if the current user has any issues in progress. If
there are any issues in progress, it will attempt to:
\begin{enumerate}
\item Create/merge a branch for the stash. 
\item Set the old Jira issue to no longer be "In Progress".
\item Switch back to trunk.
\item Change the new jira issue to be "In Progress".
\end{enumerate}

It is important to know that also switching between issues will result
in the same. The difference between assign and switch is that a jira
issue will change assignment with assign and not with switch.

\subsection{Liquibase Changes}
In the past, creating a database change was a hassle. In some cases,
multiple database platforms needed to be supported and it was
difficult to duplicate SQL. Copy-paste was rampant and led to numerous
build failures. Even with build failsafes in place, there were
problems with almost every database change because there was no way to
test beforehand.

\subsubsection{Creating a Change}

To create a database change, do the following:
\begin{lstlisting}[language=xml,caption={Create a change in liquibase}]
<customChange class="org.liquibase.change.ext.CreateResponsibility">
    <param name="template"  value="Review" />
    <param name="namespace" value="KFS-TEM" />
    <param name="name"      value="Review" />
    <param name="active"    value="Y" />
</customChange>
\begin{lstlisting}

Or...
\begin{lstlisting}[language=xml,caption={Add a system parameter}]
<customChange class="org.liquibase.change.ext.AddSystemParameter">
    <param name="application" value="Y" />
    <param name="namespace" value="KFS-TEM" />
    <param name="name" value="ENABLE_PER_DIEM_LOOKUP_LINKS_IND" />
    <param name="detailTypeCode" value="TravelAuthorization" />
    <param name="typeCode" value="CONFG" />
    <param name="description" value="Y" />
    <param name="value" value="Y" />
    <param name="constraintCode" value="Y" />
</customChange >
\end{lstlisting}

Normally, these are very database specific issues, but with some
custom refactorings we can remove that and make it this easy. Even
within projects, some may require their own refactorings that can be
added. This will be especially useful when developing test data for
integration tests.


\subsubsection{Testing a Change}

Once I have created a liquibase change, I will need to test it
somehow. If my project supports multiple platforms, I should test
against them all. Here is how we would do it. First setup the tests.

\begin{lstlisting}
mvn validate testResources
\end{lstlisting}

You may recall that the changelogs are copied during the validate
goal. I run testResources to copy my properties files to the
appropriate locations. After doing that, I should see this in
target/changelogs/update/

\begin{lstlisting}
leo@behemoth~/.workspace/kfs/release-4-0-overlay
(21:38:19) [540] ls target/changelogs/update/CM-156.xml
\end{lstlisting}

I see the changelog I created. Good. I should also see my properties
in target/test-classes/liquibase

\begin{lstlisting}
leo@behemoth~/.workspace/kfs/release-4-0-overlay
(22:02:22) [541] ls target/test-classes/liquibase/
TEM.properties          TEMNIGHTLY.properties
liquibase.properties.template
\end{lstlisting}

There you have it. Now we're ready to test.

\begin{lstlisting}
 leo@behemoth~/.workspace/kfs/release-4-0-overlay
(21:29:10) [537] mvn validate lb:test
[INFO] Scanning for projects...
[INFO]                                                                        
[INFO] ------------------------------------------------------------------------
[INFO] Building kfs 4.0M2
[INFO] ------------------------------------------------------------------------
[INFO]
[INFO] --- maven-resources-plugin:2.5:copy-resources (copy-test-changelogs) @ kfs ---
[debug] execute contextualize
[INFO] Using 'UTF-8' encoding to copy filtered resources.
[INFO] Copying 1 resource
[INFO]
[INFO] --- lb-maven-plugin:0.0.1:test (default-cli) @ kfs ---
[WARNING] Artifact with no actual file, 'org.kuali.kfs:kfs'
[WARNING] Artifact with no actual file, 'commons-lang:commons-lang'
[WARNING] Artifact with no actual file, 'com.lowagie:itext'
[WARNING] Artifact with no actual file, 'jasperreports:jasperreports'
[WARNING] Artifact with no actual file, 'org.kuali.kfs:kfs'
[WARNING] Artifact with no actual file, 'mysql:mysql-connector-java'
[WARNING] Artifact with no actual file, 'junit:junit'
[WARNING] Artifact with no actual file, 'javax.servlet:servlet-api'
[WARNING] Artifact with no actual file, 'javax.servlet:jstl'
[WARNING] Artifact with no actual file, 'taglibs:standard'
[WARNING] Artifact with no actual file, 'javax.servlet:jsp-api'
[WARNING] Artifact with no actual file, 'org.eclipse.jetty:jetty-deploy'
[WARNING] Artifact with no actual file, 'org.eclipse.jetty:jetty-jsp-2.1'
[WARNING] Artifact with no actual file, 'org.eclipse.jetty:jetty-server'
[WARNING] Artifact with no actual file, 'org.eclipse.jetty:jetty-webapp'
[WARNING] Artifact with no actual file, 'org.hamcrest:hamcrest-library'
[WARNING] Artifact with no actual file, 'org.springframework:spring-beans'
[WARNING] Artifact with no actual file, 'org.springframework:spring-context'
[WARNING] Artifact with no actual file, 'org.springframework:spring-context-support'
[WARNING] Artifact with no actual file, 'org.springframework:spring-core'
[WARNING] Artifact with no actual file, 'org.springframework:spring-jdbc'
[WARNING] Artifact with no actual file, 'org.springframework:spring-tx'
[WARNING] Artifact with no actual file, 'org.springmodules:spring-modules-ojb'
[INFO] Parsing Liquibase Properties File
[INFO]   File: /Users/leo/.workspace/kfs/release-4-0-overlay/target/test-classes/liquibase/TEM.properties
[INFO] ------------------------------------------------------------------------
[INFO] ------------------------------------------------------------------------
[WARNING] Artifact with no actual file, 'org.kuali.kfs:kfs'
[WARNING] Artifact with no actual file, 'commons-lang:commons-lang'
[WARNING] Artifact with no actual file, 'com.lowagie:itext'
[WARNING] Artifact with no actual file, 'jasperreports:jasperreports'
[WARNING] Artifact with no actual file, 'org.kuali.kfs:kfs'
[WARNING] Artifact with no actual file, 'mysql:mysql-connector-java'
[WARNING] Artifact with no actual file, 'junit:junit'
[WARNING] Artifact with no actual file, 'javax.servlet:servlet-api'
[WARNING] Artifact with no actual file, 'javax.servlet:jstl'
[WARNING] Artifact with no actual file, 'taglibs:standard'
[WARNING] Artifact with no actual file, 'javax.servlet:jsp-api'
[WARNING] Artifact with no actual file, 'org.eclipse.jetty:jetty-deploy'
[WARNING] Artifact with no actual file, 'org.eclipse.jetty:jetty-jsp-2.1'
[WARNING] Artifact with no actual file, 'org.eclipse.jetty:jetty-server'
[WARNING] Artifact with no actual file, 'org.eclipse.jetty:jetty-webapp'
[WARNING] Artifact with no actual file, 'org.hamcrest:hamcrest-library'
[WARNING] Artifact with no actual file, 'org.springframework:spring-beans'
[WARNING] Artifact with no actual file, 'org.springframework:spring-context'
[WARNING] Artifact with no actual file, 'org.springframework:spring-context-support'
[WARNING] Artifact with no actual file, 'org.springframework:spring-core'
[WARNING] Artifact with no actual file, 'org.springframework:spring-jdbc'
[WARNING] Artifact with no actual file, 'org.springframework:spring-tx'
[WARNING] Artifact with no actual file, 'org.springmodules:spring-modules-ojb'
[INFO] Parsing Liquibase Properties File
[INFO]   File: /Users/leo/.workspace/kfs/release-4-0-overlay/target/test-classes/liquibase/TEM.properties
[INFO] ------------------------------------------------------------------------
[INFO] Executing on Database: jdbc:mysql://localhost:3306/TEM
[INFO] Tagging the database
INFO 9/21/11 9:29 PM:liquibase: Successfully acquired change log lock
INFO 9/21/11 9:29 PM:liquibase: Reading from `DATABASECHANGELOG`
INFO 9/21/11 9:29 PM:liquibase: Successfully released change log lock
[INFO] Doing update
INFO 9/21/11 9:29 PM:liquibase: Successfully acquired change log lock
INFO 9/21/11 9:29 PM:liquibase: Reading from `DATABASECHANGELOG`
INFO 9/21/11 9:29 PM:liquibase: ChangeSet /Users/leo/.workspace/kfs/release-4-0-overlay/target/changelogs/update/CM-156.xml::CM-156-1::kuali (generated) ran successfully in 37ms
INFO 9/21/11 9:29 PM:liquibase: Successfully released change log lock
[INFO] Doing rollback
INFO 9/21/11 9:29 PM:liquibase: Successfully acquired change log lock
INFO 9/21/11 9:29 PM:liquibase: Rolling Back Changeset:/Users/leo/.workspace/kfs/release-4-0-overlay/target/changelogs/update/CM-156.xml::CM-156-1::kuali (generated)::(Checksum: 3:85a5e658332342fba8b60df3a29fc393)
INFO 9/21/11 9:29 PM:liquibase: Successfully released change log lock
INFO 9/21/11 9:29 PM:liquibase: Successfully released change log lock
[INFO] ------------------------------------------------------------------------
[INFO]
[INFO] Parsing Liquibase Properties File
[INFO]   File: /Users/leo/.workspace/kfs/release-4-0-overlay/target/test-classes/liquibase/TEMNIGHTLY.properties
[INFO] ------------------------------------------------------------------------
[INFO] ------------------------------------------------------------------------
[WARNING] Artifact with no actual file, 'org.kuali.kfs:kfs'
[WARNING] Artifact with no actual file, 'commons-lang:commons-lang'
[WARNING] Artifact with no actual file, 'com.lowagie:itext'
[WARNING] Artifact with no actual file, 'jasperreports:jasperreports'
[WARNING] Artifact with no actual file, 'org.kuali.kfs:kfs'
[WARNING] Artifact with no actual file, 'mysql:mysql-connector-java'
[WARNING] Artifact with no actual file, 'junit:junit'
[WARNING] Artifact with no actual file, 'javax.servlet:servlet-api'
[WARNING] Artifact with no actual file, 'javax.servlet:jstl'
[WARNING] Artifact with no actual file, 'taglibs:standard'
[WARNING] Artifact with no actual file, 'javax.servlet:jsp-api'
[WARNING] Artifact with no actual file, 'org.eclipse.jetty:jetty-deploy'
[WARNING] Artifact with no actual file, 'org.eclipse.jetty:jetty-jsp-2.1'
[WARNING] Artifact with no actual file, 'org.eclipse.jetty:jetty-server'
[WARNING] Artifact with no actual file, 'org.eclipse.jetty:jetty-webapp'
[WARNING] Artifact with no actual file, 'org.hamcrest:hamcrest-library'
[WARNING] Artifact with no actual file, 'org.springframework:spring-beans'
[WARNING] Artifact with no actual file, 'org.springframework:spring-context'
[WARNING] Artifact with no actual file, 'org.springframework:spring-context-support'
[WARNING] Artifact with no actual file, 'org.springframework:spring-core'
[WARNING] Artifact with no actual file, 'org.springframework:spring-jdbc'
[WARNING] Artifact with no actual file, 'org.springframework:spring-tx'
[WARNING] Artifact with no actual file, 'org.springmodules:spring-modules-ojb'
[INFO] Parsing Liquibase Properties File
[INFO]   File: /Users/leo/.workspace/kfs/release-4-0-overlay/target/test-classes/liquibase/TEMNIGHTLY.properties
[INFO] ------------------------------------------------------------------------
[INFO] Executing on Database: jdbc:oracle:thin:@heisenberg.rsmart.com:1521:KFS
[INFO] Tagging the database
INFO 9/21/11 9:29 PM:liquibase: Successfully acquired change log lock
INFO 9/21/11 9:29 PM:liquibase: Reading from DATABASECHANGELOG
INFO 9/21/11 9:29 PM:liquibase: Successfully released change log lock
[INFO] Doing update
INFO 9/21/11 9:29 PM:liquibase: Successfully acquired change log lock
INFO 9/21/11 9:29 PM:liquibase: Reading from DATABASECHANGELOG
INFO 9/21/11 9:29 PM:liquibase: ChangeSet /Users/leo/.workspace/kfs/release-4-0-overlay/target/changelogs/update/CM-156.xml::CM-156-1::kuali (generated) ran successfully in 205ms
INFO 9/21/11 9:29 PM:liquibase: Successfully released change log lock
[INFO] Doing rollback
INFO 9/21/11 9:29 PM:liquibase: Successfully acquired change log lock
INFO 9/21/11 9:29 PM:liquibase: Rolling Back Changeset:/Users/leo/.workspace/kfs/release-4-0-overlay/target/changelogs/update/CM-156.xml::CM-156-1::kuali (generated)::(Checksum: 3:85a5e658332342fba8b60df3a29fc393)
INFO 9/21/11 9:29 PM:liquibase: Successfully released change log lock
INFO 9/21/11 9:29 PM:liquibase: Successfully released change log lock
[INFO] ------------------------------------------------------------------------
[INFO]
[INFO] ------------------------------------------------------------------------
[INFO] BUILD SUCCESS
[INFO] ------------------------------------------------------------------------
[INFO] Total time: 28.920s
[INFO] Finished at: Wed Sep 21 21:29:49 MST 2011
[INFO] Final Memory: 11M/262M
[INFO] ------------------------------------------------------------------------
leo@behemoth~/.workspace/kfs/release-4-0-overlay
(21:29:49) [538]
\end{lstlisting}


\section{Setup}

Much of the above implies configuration has been made in the local
\emph{\$HOME/.m2/settings.xml} to include the following information:
\begin{itemize}
\item SVN credentials
\item Jira credentials
\item Jira URL
\end{itemize}

\subsection{Project Structure}
Part of simplifying the development process is a convention/standard
of project structure. Most of this will be implied by the maven
project archetype.

\noindent \begin{tabular}{l|p{7.5cm}}
Path & Description \\
\hline \\
$settings.xml$ & Project-specific maven settings. This is a maven
convention. This settings.xml will get checked in with the project. It
is not the same as \emph{\$HOME/.m2/settings.xml}. \\
\hline \\
$.m2/repository$ & Project-specific maven repository. Projects will no
longer share a maven repository. Each project/environment will get its
own local repo. This will prevent repos from interfering for
developers. \\
\hline \\
$scripts/changesets/latest/$ & Liquibase scripts for building a database
from scratch \\
\hline \\
$scripts/changesets/latest/cst/$ & Liquibase scripts for building a database
from scratch. Constraints. \\
\hline \\
 $scripts/changesets/latest/dat/$ & Liquibase scripts for building a database
from scratch. Data. \\
\hline \\
 $scripts/changesets/latest/idx/$ & Liquibase scripts for building a database
from scratch. Indexes. \\
\hline \\
$scripts/changesets/latest/seq/$ & Liquibase scripts for building a database
from scratch. Sequences. \\
\hline \\
$ scripts/changesets/latest/tab/$ & Liquibase scripts for building a database
from scratch. Tables. \\
\hline \\
$ scripts/changesets/latest/vw/$ & Liquibase scripts for building a database
from scratch. Views. \\
\hline \\
\end{tabular}
\noindent \begin{tabular}{l|p{7.5cm}}
Path & Description \\
\hline \\
$scripts/changesets/update/$ & scripts is the normal path for database
scripts. I added a subset to imply a liquibase changeset. The naming
convention is JIRA-ISSUE-111-svnrevision.xml. svn revision is included
in the name because there can be numerious changes to an issue. These
need to be independent of the issue itself unless each change strictly
implied a new jira issue. That is just too unreasonable.\\
\hline \\
 $changelists/$ & not to be confused with liquibase changesets, these are
the files which are part of the jira issue and files that are impacted
by the release. \\
\hline \\
$it/$ & Maven projects should be modular including a module for
integration tests\\
\hline \\
$web/$ & Maven projects should be modular including a module for
web \\
\hline \\
$core/$ & Maven projects should be modular including a module for
core code 
\end{tabular}

\subsection{Conventions}
I cannot think of any I'd like to add at this point

\end{document}

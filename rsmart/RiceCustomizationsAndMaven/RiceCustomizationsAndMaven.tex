\documentclass[12pt,notitlepage]{article}
\author{Leo Przybylski}
\usepackage{minted}
%%\usepackage{etoolbox}
\usepackage{fancyhdr}
\usepackage{graphicx}
\usepackage{hyperref}
%%\usepackage[usenames,dvipsnames]{color}

\title{\includegraphics[width=0.75\textwidth]{../../images/rsmart_base.png}\\\includegraphics[width=0.70\textwidth]{../../images/kuali_base.png}\\rSmart
Rice Customizations and Maven}
\pagestyle{fancy}
\fancyhead{} % Clear all header fields 
\fancyhead[OL]{\sectionmark}
\fancyhead[OR]{\includegraphics[height=26pt]{../../images/rsmart_base.png}}% 
\fancyhead[ER]{\sectionmark}
\fancyhead[EL]{\includegraphics[height=26pt]{../../images/rsmart_base.png}}% 
%%\definecolor{ubergray}{RGB}{245,245,245}
%%\hypersetup{colorlinks}
%%\lstset{basicstyle=\scriptsize,
%%  backgroundcolor=\color{ubergray},
%%  breaklines=true,
%%  frame=single,
%%  includerangemarker=false}
\begin{document}
\maketitle
\tableofcontents

\abstract{When an institution wants to implement customizations and/or fixes from another rice distribution, how do these customizations get pushed out to other projects? The best way is through maven and the institution's Maven repository. The trouble is that Rice is a modular project where modules depend on the parent. It is important to keep the following in mind: \begin{itemize}
\item Original project parents
\item Do not use the org.kuali groupId
\end{itemize}
This document is about how to go about this.}

\section{Overview}

The following is the module layout for Rice:
{\ttfamily \begin{list}{|->}{\setlength{\leftmargin}{1em}}
  \item rice
    
    \begin{list}{|->}{\setlength{\leftmargin}{1em}}
    \item[\textbackslash->] client-contrib 
    \item config
      \begin{list}{|->}{\setlength{\leftmargin}{0.5em}}
      \item[\textbackslash->] access 
      \item archetype
      \item checkstyle
      \item deploy
      \item ide
      \end{list}
    \item core
      \begin{list}{|->}{\setlength{\leftmargin}{0.5em}}
      \item[\textbackslash->] api 
      \item framework
      \item impl
      \item web
      \end{list}
    \item core-service
      \begin{list}{|->}{\setlength{\leftmargin}{0.5em}}
      \item[\textbackslash->] api 
    \item framework
    \item impl
    \item web
    \end{list}
  \item db
    \begin{list}{|->}{\setlength{\leftmargin}{0.5em}}
    \item[\textbackslash->] impex
      \begin{list}{|->}{\setlength{\leftmargin}{0.5em}}
      \item[\textbackslash->] client
        \begin{list}{|->}{\setlength{\leftmargin}{0.5em}}
        \item[\textbackslash->] bootstrap
        \item demo
        \end{list}
      \item master
      \item server
        \begin{list}{|->}{\setlength{\leftmargin}{0.5em}}
        \item[\textbackslash->] bootstrap
        \item demo
        \end{list}
      \end{list}
    \item sql
    \end{list}
    
  \item development-tools
  \item dist
  \item edl
    \begin{list}{|->}{\setlength{\leftmargin}{0.5em}}
    \item[\textbackslash->] framework
      \item impl
    \end{list}
  \item impl
  \item it
    \begin{list}{|->}{\setlength{\leftmargin}{0.5em}}
    \item[\textbackslash->] config
      \item core
      \item edl
      \item impl
      \item internal-tools
      \item kcb
      \item ken
      \item kew
      \item kim
      \item krad
      \item krms
      \item ksb
      \item location
      \item vc
    \end{list}
  \item ken
    \begin{list}{|->}{\setlength{\leftmargin}{0.5em}}
    \item[\textbackslash->] api
    \end{list}
  \item kew
    \begin{list}{|->}{\setlength{\leftmargin}{0.5em}}
    \item[\textbackslash->] api
      \item framework
      \item impl
    \end{list}
  \item kim
    \begin{list}{|->}{\setlength{\leftmargin}{0.5em}}
    \item[\textbackslash->] kim-api
    \item kim-framework
    \item kim-impl
    \item kim-ldap
    \end{list} 
  \item kns
  \item krad
    \begin{list}{|->}{\setlength{\leftmargin}{0.5em}}
    \item[\textbackslash->] krad-app-framework
    \item krad-web-framework
    \end{list}
  \item krms
    \begin{list}{|->}{\setlength{\leftmargin}{0.5em}}
    \item[\textbackslash->] api
    \item framework
    \item impl
    \end{list}
  \item ksb
    \begin{list}{|->}{\setlength{\leftmargin}{0.5em}}
    \item[\textbackslash->] client-impl
    \item server-impl
    \end{list} 
  \item location
  \item sampleapp
  \item serviceregistry
  \item serviceregistry
  \item standalone
  \item web
  \end{list}
\end{list}}

Each module is a child that has a Maven parent back to the another module closer to the root parent. For example, \texttt{impl}'s parent is:
  \begin{minted}[fontsize=\footnotesize]{xml}
<project xmlns="http://maven.apache.org/POM/4.0.0" 
     xmlns:xsi="http://www.w3.org/2001/XMLSchema-instance" 
     xsi:schemaLocation="http://maven.apache.org/POM/4.0.0 
                         http://maven.apache.org/maven-v4_0_0.xsd">
  <name>Rice Implementation</name>
  <modelVersion>4.0.0</modelVersion>
  <parent>
    <groupId>org.kuali.rice</groupId>
    <artifactId>rice</artifactId>
    <version>2.3.0-SNAPSHOT</version>
  </parent>
  <artifactId>rice-impl</artifactId>
...
...
</project>
  \end{minted}

\end{document}
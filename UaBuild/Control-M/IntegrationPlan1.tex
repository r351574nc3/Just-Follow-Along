\documentclass[12pt,notitlepage]{article}
\author{Leo Przybylski \\
\texttt{przybyls@arizona.edu}}
\usepackage{listings}
\usepackage{color}
\usepackage{graphicx}

\title{Control-M Integration Plan}

\begin{document}
\tableofcontents
\maketitle

\section{Overview}

Control-M can handle scheduling for remote services that reside on KFS. Control-M manages job registration,
metadata, and scheduling internally via the agent. It will invoke jobs on KFS remotely via web services call 
to the Kuali Rice ESB. A web service on the Rice ESB can integrate with KFS Quartz jobs defined in Spring. This way
scheduling is deferred to Control-M while the jobs themselves are developed and implemented for Quartz using the 
Quartz API and Spring IOC.

\section{Control-M Web Services}

Control-M BPI module has presets for communicating with various J2EE appliction servers. It also has 
a generic Web Services interface where input/output parameters can be specified for a web service call.


\subsection{Control-M Callbacks}
If necessary, the Rice ESB registered web service can make callbacks to the Control-M BPI Control Module via
\verb|EMXmlInvoker| of the Control-M EM API.

\section{Kuali Enterprise Service Bus}
The Kuali Rice ESB can be used to expose services to external applications via web services. A service
published on the Rice ESB can be given access to KFS Quartz jobs registered internally within Spring.

\section{Integration Tasks}
\subsection{At Control-M BPI Control Module}
Define web services parameters for job processing:
\begin{itemize}
\item Specify the account (an account is a logical name that encapsulates a repository of WSDLs) 
\item Specify the business, service, and operation to invoke the full hierarchy (all information is populated automatically without manual typing or configuration)
\item Specify input/output parameters (dynamically displays a list of all input and output parameters available for the selected operation)
\end{itemize}

Since Quartz no longer handles scheduling for batch jobs, each job will have to be added to the agent with BPI Control Module. Web services
will be required to be configured for each.

\subsection{On Rice ESB}
\begin{itemize}
  \item Create WSDL for a web service Control-M will communicate with on the Rice ESB. Input/Output
    parameters can be configured in the Control BPI Control Module, so it isn't necessary to create a layer to handle input/output
    parameters.
  \item Develop an exposed web service registered on the Rice ESB
    \begin{itemize}
      \item Registers Quartz jobs defined in KFS through Spring IOC
      \item Exposed for remote call from the Control-M BPI Control Module
      \item Accepts parameters and responds with output information
      \item Reports success or failure
      \item Generates log data
    \end{itemize}
\end{itemize}

\end{document}

\documentclass[12pt,notitlepage]{article}
\author{Leo Przybylski \\
\texttt{przybyls@arizona.edu}}
\usepackage{listings}
\usepackage{color}
\usepackage{graphicx}

\title{KFS RPM Packaging Proposal}

\begin{document}
\maketitle

\lstset{basicstyle=\small,
  breaklines=true,
  includerangemarker=false}

\section{Overview}
The objective is to standardize deployment of the application an possible configurations across 
platforms and environments. Currently, we use a non-standard tool for deploying artifacts and exposing
login credentials via public/private key across several environments. Deployment forces taking down an 
application server for a large amount of time unnecessarily and preforms numerous redundant tasks against
the environment. 

The following improvements can be made through RPM Packages:

\begin{itemize}
  \item Packages allow better accuracy of what is performed as well as reporting on previous installations. Also, RPM packages take advantage of previously executed deferred processing to do this. 
  \item Previously built nature of packages, allows a gain in 
    \begin{itemize}
      \item compilation of java source 
      \item XML-to-SQL source code generation
      \item file transfers
    \end{itemize}
  \item Dramatically increases the speed and thus reduces downtime
  \item RPM packages are also an established, well-known standard in distributing software on Redhat Linux distributions.
\end{itemize}

\section{RPM Packages}
RPM stands for ``Redhat Package Manager.'' RPM Packages are the standard for Redhat Linux distributions.
It defines a standard for dependency management and dependency packaging.

\subsection{kfs.spec}
When building RPM packages, a spec file is required to instruct RPM how to create the packages. Information like
dependencies, file locations, pre-processing and post-processing scripts, etc\ldots are described within it.

\section{Using RPM Packages}
When installing RPM packages, the preferred command is the following:
\begin{lstlisting}
rpm -Uvh --force /mosaic/data/KITT/RPMS/noarch/kfs-3.0-dmo.1196.26.noarch.rpm
\end{lstlisting}

\begin{description}
  \item[-Uvh] Upgrade showing verbose output and hash (\#) marks for progress. Always use upgrade
\end{description}

It can be determined what packages are running on an installation via:

\begin{lstlisting}
tomcat@uaz-kf-a02:~> rpm -qa | grep kfs
kfs-workflow-3.0-dev.1196.29
kfs-3.0-dev.1196.29
kfs-encrypt-3.0-dev.1196.29
\end{lstlisting}

The following can be used to report what workflow XML files were ingested:

\lstset{basicstyle=\tiny,
  breaklines=true,
  includerangemarker=false}
\begin{lstlisting}
tomcat@uaz-kf-a02:/mosaic/data/KITT/RPMS/noarch> rpm -ql kfs-workflow-3.0-dev.1196.29
/home/tomcat/app/work/dev/kfs/staging/workflow/ingestion
/home/tomcat/app/work/dev/kfs/staging/workflow/ingestion/CheckReconciliationMaintenanceDocument.xml
/home/tomcat/app/work/dev/kfs/staging/workflow/ingestion/DisbursementVoucherBatchDefaultMaintenanceDocument.xml
/home/tomcat/app/work/dev/kfs/staging/workflow/ingestion/FACostSubCategoryMaintenanceDocument.xml
/home/tomcat/app/work/dev/kfs/staging/workflow/ingestion/PRJESetMaintenance.xml
/home/tomcat/app/work/dev/kfs/staging/workflow/ingestion/PRJETypeMaintenance.xml
/home/tomcat/app/work/dev/kfs/staging/workflow/ingestion/PayeeMaintenanceDocument.xml
/home/tomcat/app/work/dev/kfs/staging/workflow/ingestion/PayerMaintenanceDocument.xml
/home/tomcat/app/work/dev/kfs/staging/workflow/ingestion/PaymentMaintenanceDocument.xml
/home/tomcat/app/work/dev/kfs/staging/workflow/ingestion/ProcurementCardDocument.xml
/home/tomcat/app/work/dev/kfs/staging/workflow/ingestion/ProcurementCardHolderDetailMaintenanceDocument.xml
/home/tomcat/app/work/dev/kfs/staging/workflow/ingestion/PurchaseOrderBatchDefaultMaintenanceDocument.xml
/home/tomcat/app/work/dev/kfs/staging/workflow/ingestion/RouteCodeMaintenanceDocument.xml
/home/tomcat/app/work/dev/kfs/staging/workflow/ingestion/SourceOfFundsMaintenanceDocument.xml
\end{lstlisting}

Information about the RPM can be gotten like this:

\begin{lstlisting}
tomcat@uaz-kf-a02:/mosaic/data/KITT/RPMS/noarch> rpm -qi kfs-encrypt-3.0-dev.1196.29
Name        : kfs-encrypt                  Relocations: (not relocatable)
Version     : 3.0                               Vendor: (none)
Release     : dev.1196.29                   Build Date: Tue 07 Jul 2009 01:14:41 AM MST
Install Date: Wed 08 Jul 2009 10:44:58 AM MST      Build Host: uaz-kf-a00.mosaic.arizona.edu
Group       : System/Base                   Source RPM: kfs-3.0-dev.1196.29.src.rpm
Size        : 279                              License: EPL
Signature   : (none)
Packager    : przybyls@arizona.edu
Summary     : Kuali Financial System Field Encryption
Description :
Mosaic KFS dev Environment Field Encryption Data based on rSmart Vendor build 1196
with KITT modification set 29
\end{lstlisting}

\subsection{kitt-tools}
\subsubsection{Installing}
Run the following command as the \emph{tomcat} user

\lstset{basicstyle=\small,
  breaklines=true,
  includerangemarker=false}
\begin{lstlisting}
rpm -Uvh --force --nodeps --dbpath /home/tomcat/app/rpm /mosaic/data/KITT/RPMS/noarch/kitt-tools-1.0-1.noarch.rpm 
\end{lstlisting}

\begin{description}
  \item[-Uvh] Upgrade showing verbose output and hash (\#) marks for progress
  \item[--force] A macro for upgrading both files and package names in the database
  \item[--nodeps] Initially, the database will be empty, so none of the dependencies are expected to pass. This
    is fixed by installing kitt-tools, so it is necessary when installing them.
  \item[--dbpath] Initially, the dbpath is set to /var/lib/rpm which is not what we want. This will be
    fixed upon installing kitt-tools, but is necessary for the first installation.
  \item[/mosaic/data/KITT/RPMS/noarch/kitt-tools-1.0-1.noarch.rpm] is the path of the RPM package. 
    \verb|/mosaic/data/kitt/RPMS/noarch| are where all RPM packages are stored that are created by
    KITT.
\end{description}
\section{KITT Packages}
Below is a list of packages that are created through the DEV Continuous Integration process.

\lstset{basicstyle=\tiny,
  breaklines=true,
  includerangemarker=false}

\begin{lstlisting}
[przybyls@uaz-kf-a00 ~]$ cd /mosaic/data/KITT/RPMS/noarch/
[
[przybyls@uaz-kf-a00 noarch]$ ls *1196.23*
kfs-3.0-cfg.1196.23.noarch.rpm             kfs-encrypt-3.0-cfg.1196.23.noarch.rpm
kfs-3.0-cnv.1196.23.noarch.rpm             kfs-encrypt-3.0-cnv.1196.23.noarch.rpm
kfs-3.0-dev.1196.23.noarch.rpm             kfs-encrypt-3.0-dev.1196.23.noarch.rpm
kfs-3.0-trn.1196.23.noarch.rpm             kfs-encrypt-3.0-trn.1196.23.noarch.rpm
kfs-3.0-tst.1196.23.noarch.rpm             kfs-encrypt-3.0-tst.1196.23.noarch.rpm
kfs-bootstrap-3.0-cfg.1196.23.noarch.rpm   kfs-kittdb-3.0-cfg.1196.23.noarch.rpm
kfs-bootstrap-3.0-cnv.1196.23.noarch.rpm   kfs-kittdb-3.0-cnv.1196.23.noarch.rpm
kfs-bootstrap-3.0-dev.1196.23.noarch.rpm   kfs-kittdb-3.0-dev.1196.23.noarch.rpm
kfs-bootstrap-3.0-trn.1196.23.noarch.rpm   kfs-kittdb-3.0-trn.1196.23.noarch.rpm
kfs-bootstrap-3.0-tst.1196.23.noarch.rpm   kfs-kittdb-3.0-tst.1196.23.noarch.rpm
kfs-conversion-3.0-cfg.1196.23.noarch.rpm  kfs-rsmartdb-3.0-cfg.1196.23.noarch.rpm
kfs-conversion-3.0-cnv.1196.23.noarch.rpm  kfs-rsmartdb-3.0-cnv.1196.23.noarch.rpm
kfs-conversion-3.0-dev.1196.23.noarch.rpm  kfs-rsmartdb-3.0-dev.1196.23.noarch.rpm
kfs-conversion-3.0-trn.1196.23.noarch.rpm  kfs-rsmartdb-3.0-trn.1196.23.noarch.rpm
kfs-conversion-3.0-tst.1196.23.noarch.rpm  kfs-rsmartdb-3.0-tst.1196.23.noarch.rpm
kfs-demodata-3.0-cfg.1196.23.noarch.rpm    kfs-workflow-3.0-cfg.1196.23.noarch.rpm
kfs-demodata-3.0-cnv.1196.23.noarch.rpm    kfs-workflow-3.0-cnv.1196.23.noarch.rpm
kfs-demodata-3.0-dev.1196.23.noarch.rpm    kfs-workflow-3.0-dev.1196.23.noarch.rpm
kfs-demodata-3.0-trn.1196.23.noarch.rpm    kfs-workflow-3.0-trn.1196.23.noarch.rpm
kfs-demodata-3.0-tst.1196.23.noarch.rpm    kfs-workflow-3.0-tst.1196.23.noarch.rpm
\end{lstlisting}

\lstset{basicstyle=\small,
  breaklines=true,
  includerangemarker=false}

\subsection{Naming}
The convention for naming of packages is as follows:
\verb|{application name}|-\verb|{version}|-\verb|{environment}|-\verb|{build number}|.
\verb|{kitt build}|.\verb|{architecture}|.rpm

\begin{description}
  \item[application name] can be kfs, kc, or rice. It represents the name of the application 
    being packaged. Within each application, there may be subpackages. Here is a list of sub
    packages for KFS.
    \begin{description}
      \item[bootstrap] is the bootstrap dataset. This package will install/run SQL scripts required
        for bootstrapping the database configured for this package. Consists of only
        \verb|CREATE TABLE|, \verb|INSERT|, and \verb|ALTER TABLE| SQL statements.
      \item[conversion] is the conversion dataset. This package will install/run SQL scripts required
        for adding conversion data to the database configured for this package. Consists of only
        \verb|CREATE TABLE|, \verb|INSERT|, and \verb|ALTER TABLE| SQL statements.
      \item[demodata] is the ``demo'' dataset. This package will install/run SQL scripts required for
        adding demo data to the database configured for this package. Consists of only
        \verb|CREATE TABLE|, \verb|INSERT|, and \verb|ALTER TABLE| SQL statements.
      \item[encryption] encrypts the database fields configured for this environment
      \item[kittdb] is the KITT dataset. Only install this if you intend to use KITT modifications. At
        the time of writing this document, only the DEV environment does this. Consists of only
        \verb|CREATE TABLE|, \verb|INSERT|, and \verb|ALTER TABLE| SQL statements.
      \item[rsmartdb] is the rSmart dataset. Only install this if you intend to use rSmart modifications.
        Consists of only \verb|CREATE TABLE|, \verb|INSERT|, and \verb|ALTER TABLE| SQL statements.
      \item[workflow] contains all of the workflow XML that needs to be ingested prior to application use.
        Relies on the ingester to pick these up prior to application start. Consists of only
        \verb|CREATE TABLE|, \verb|INSERT|, and \verb|ALTER TABLE| SQL statements.
    \end{description}
  \item[version] of the application. For KFS at this time, it is 3.0.
  \item[environment] being packaged for. The configuration files are tightly couple to the
    environment, so each build gets a specific environment it's tied to.
  \item[build number] the Kuali Community build number for the application. It is used to identify
    which build is being used.
  \item[kitt build] the build number that the UA Kuali Implementation Technical Team uses to version 
    its modifications.
  \item[architecture] will always be ``noarch'' because Kuali is intended to run architecture independent.
\end{description}
\end{document}

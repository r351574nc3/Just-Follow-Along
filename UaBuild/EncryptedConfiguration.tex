\documentclass[12pt,notitlepage]{article}
\author{Leo Przybylski \\
\texttt{przybyls@arizona.edu}}
\usepackage{listings}
\usepackage{color}
\usepackage{graphicx}

\title{Encrypted Password Configuration Proposal}

\begin{document}
\maketitle

\lstset{basicstyle=\small,
  breaklines=true,
  includerangemarker=false}

\section{Overview}
Configuration files are separated by environment. Each environment gets its own configuration, but
usually only credentials are what differs. These environment specific configurations cannot be added
to SVN or any kind of version control because there are credentials in them.

This leads a problem that has caused a number of headaches. That is that whenever a change occurs to
these configuration files, they have to be propogated to the environments. Usually, the change is tested
in DEV and passed on, but the human factor can leave some elements missing or mispelled. Further, there
is no change record to determine who, when, or whether at all a configuration was updated with the 
required change.

\subsection{Configuration Files in SVN}
To alleviate the aforementioned issues, the simplest solution is to store configurations in SVN; however, 
there is still the matter of storing credentials. The question arises, ``How can we store the configuration
information entirely without having to worry about insecure credentials and while maintaining configurations
independent by environment?'' This can be done with public/private key encryption.

\subsection{Public/Private Key Encryption}
With public/private key encryption, only the encrypted data is stored in SVN while the public/private key information
is stored within the environment. With private key being the only means to unlock the encrypted credentials
residing within the environment, it can be considered that the environment still holds the credentials.

\section{Steps}
In order to enable encryption on the environment, the \emph{kitt-tools} package needs to be installed on the
environment. This will add a script for generating the RSA public/private keys.

\begin{lstlisting}
rpm -Uvh --force --nodeps --dbpath /home/tomcat/app/rpm /mosaic/data/KITT/RPMS/noarch/kitt-tools-1.0-1.noarch.rpm 
\end{lstlisting}

\begin{description}
  \item[-Uvh] Upgrade showing verbose output and hash (\#) marks for progress
  \item[--force] A macro for upgrading both files and package names in the database
  \item[--nodeps] Initially, the database will be empty, so none of the dependencies are expected to pass. This
    is fixed by installing kitt-tools, so it is necessary when installing them.
  \item[--dbpath] Initially, the dbpath is set to /var/lib/rpm which is not what we want. This will be
    fixed upon installing kitt-tools, but is necessary for the first installation.
  \item[/mosaic/data/KITT/RPMS/noarch/kitt-tools-1.0-1.noarch.rpm] is the path of the RPM package. 
    \verb|/mosaic/data/kitt/RPMS/noarch| are where all RPM packages are stored that are created by
    KITT.
\end{description}

\subsection{Creating the Key}
The public/private key pairs need to be created on the application server. This can be done with the \emph{kitt-tools}
by executing:

\begin{lstlisting}
tomcat@uaz-kf-a02:~> kitt-tools/bin/rsaGenerateKey.sh 
Buildfile: /dev/stdin

run:
     [java] Generating Rivest, Shamir, and Adleman 4096 Keypair...
     [java] ...done generating Keypair!
     [java] Saving private key...
     [java] ...saving public key...
     [java] ...done generating keys!

BUILD SUCCESSFUL
Total time: 1 minute 16 seconds
\end{lstlisting}

Two files are created \verb|/home/tomcat/.ssh/kuali_rsa| and \verb|/home/tomcat/.ssh/kuali_rsa.pub|. These
will be used for encryption/decryption. The public key is used for encryption and the private key
is used for decryption. The private key will need to reside in the build system while the public key
can stay on the application server. It will be used in the next step for encrypting the configuration.

\subsection{Encrypting the Configuration}
Copy the configuration file (ie., dev.properties) to the application server. Run the following command:

\begin{lstlisting}
tomcat@uaz-kf-a02:~> kitt-tools/bin/rsaEncrypt.sh ~/dev.properties 
Buildfile: /dev/stdin

run:
     [echo] Using properties file /home/tomcat/dev.properties
\end{lstlisting}

Use the full path of dev.properties because pwd and \verb|./| seem to point to \verb|/dev| on the application servers.

dev.properties will be overwritten. It may look something like this:

\begin{lstlisting}
datasource.sid=UAZKFDEV
datasource.username=kulowner
encrypted.password=GqgxNSaLktlZH3upMIBbRNQE5k93u/gO1Z3mhR7ppvsiGASTi8AWUCrsafSFt
K0CI5dSUQxb0JeTE7wBGu5nghZ6AwqHN9pzjdQ6s6uP0Ph9y8YkfbmvuaJ6uqPhwnXfced/sYt6YhIT9
RR+VeSlByuO2sSyJ4PQY+Y3+UzpBbuCNzgsbIou7ivwiuac0RIQqf4OftJBhv7lPbyPQFTIP0vHAKUoE
BUoBSIgVPtC2jM5433yADPRWYQxK9cNtHjEZBqtNepxj8pqqPOXl1WL5tvuyZqXGBvJ7olX3+CkEvjcK
f5lO1lwFfztX3Q9C9Rk1CCWXcitQ3L1HeFdb9TjonvhTE+cZl91JEp7W6xIzmIdPPeK5y0WtFvmEdOgH
uA/yURwkug8mNzy5HBuup3bsx57fv4X40/G6kcFgkh2iV6sm1S3Nmq1z4VDYireXJeZcx29i5mkRYWLq
M19bWno6Y6rq8/RafWErqIkT6V6mTpMF+bUQ/nBbyHLAQw1G+TX2htisaSCRYSSrDWawZJSLMEENBc1j
eoQZTq9cLjPjoa9Be6y19gwX1sSmcw9xtYvCNw4xsAPER0RYYWf6uv75ebrZwwq+qy894xlf03n/mi7M
3albLBqMb0U6AJp6uK1OBOoqB4xESY4aWG1C1QykobhB0zPoB6cUeUL//Ou+do=
oracle.datasource.url=jdbc:oracle:thin:@uaz-kf-d02.mosaic.arizona.edu:1521:UAZKFDEV
batch.mailing.list=kuali+dev@email.arizona.edu
post.target.server=uaz-kf-a02.mosaic.arizona.edu
shared.external.build.properties=${shared.external.properties}
appserver.name=uaz-kf-a02.mosaic.arizona.edu:8240
appserver.url=http://uaz-kf-a02.mosaic.arizona.edu:8240/
cas.application=cas
externalizable.static.content.url=https://pa-dev.mosaic.arizona.edu:8443/kfs/static

\end{lstlisting}

\subsection{Decrypting the Configuration}
Decryption happens at build time. The private key resides in the \verb|.ssh/| directory
associated by environment name. For example, \verb|/home/tomcat/.ssh/kfs_dev/| is the file
path for the dev environment private key. This key is used by an automation tool when the 
environment is built to decrypt the password.

\end{document}

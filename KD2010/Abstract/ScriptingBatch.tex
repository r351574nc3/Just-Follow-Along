\documentclass[12pt,notitlepage]{article}
\author{Leo Przybylski \\
\texttt{przybyls@arizona.edu}}
\usepackage{listings}
\usepackage{color}
\usepackage{graphicx}

\title{Scripting Batch Processing with Web Services}


\begin{document}
\maketitle

\abstract{One of the necessities implementing schools have is to remotely 
execute batch processes. Many institutions already have existing enterprise
scheduling systems that are far more robust than Quartz which is shipped with
KFS. The cases for this are when implementing institutions use a third-party
scheduling system (Peoplesoft or BMC Control-M). Quartz scheduling is often
described as being ``simple'' and ``not an enterprise level scheduling system.''
It is almost certain that institutions will want to put in place something more
robust or may even already have a campus wide enterprise scheduling system.

This is an informative talk on how to connect an enterprise scheduling system with KFS.

Attendees will see examples and scenarios of how to integrate using methods like
shell-scripting and web-services.

Takeaways for attendees are that they will:
\begin{itemize}
  \item learn about caveats of normal shell-scripting of batch processes
  \item experience caveats of running batch processing during scheduled maintenance periods
  \item learn of creating a simple web service that isn't published to KSB for executing isolated batch processes.
  \item learn how to create a client for communicating with the batch wsdl
\end{itemize}

}
\end{document}

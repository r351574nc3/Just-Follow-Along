\documentclass[12pt]{report}
\author{Leo Przybylski}
\usepackage{fancyhdr}
\usepackage{graphicx}
\usepackage{hyperref}
\usepackage{listings}
\usepackage{minted}

\title{\includegraphics[width=0.70\textwidth]{../images/kuali_base.png}\\Customization Portal Technical Proposal}
\pagestyle{fancy}
\fancyhead{} % Clear all header fields 
\fancyhead[OL]{\sectionmark}
\fancyhead[OR]{\includegraphics[height=26pt]{../images/kuali_base.png}}% 
\fancyhead[ER]{\sectionmark}
\fancyhead[EL]{\includegraphics[height=26pt]{../images/kuali_base.png}}% 
\hypersetup{colorlinks}
\lstset{basicstyle=\scriptsize,
  backgroundcolor=\color{ubergray},
  breaklines=true,
  frame=single,
  includerangemarker=false}
\begin{document}
\maketitle
\tableofcontents

\abstract{This document outlines some technical details about how a developer or configuraiton manager might
interact with and use the Customization Portal. Part of what is included is how to configure and use archetypes, 
setup Maven dependencies, and configure Tomcat.}

\section{Overview}

This is a proposal to streamline customizations by providing:
\begin{itemize}
  \item A new customization methodology.
  \item A well-defined structure for customization projects.
  \item A pathway to continuous delivery of customizations.
  \item A process for assuring quality of customizations.
  \item A process that builds, packages, and releases customizations.
  \item A portal for locating customizations by feature, issue, or module.
  \item A community for maintaining customizations and promoting contributions back to the Kuali Foundation.
  \item A singular repository for locating and housing customizations.
\end{itemize}

\subsection{About Customizations}

Customizations are changes to original source code of Kuali Foundation projects implemented at an institution. For example, an
alteration in the field label of a Capital Asset Managment document within KFS would be considered a customization. The addition of
a field to a table or form within Kuali Rice would also be considered a customization. Customizations can be either:
\begin{itemize}
  \item A new feature.
  \item An enhancement.
  \item A bug fix.
\end{itemize}

Basically, anything that requires a rebuild of the software from its original, out-of-the box source code, and then requires the
source code to be maintained in a VCS would be considered a customization.

\subsection{The Problem}

Currently, customizations and contributions have no structural order or consistent method of integration. Developers can build
customizations in any number of ways. Most importantly, developers can develop customizations in ways that make it difficult to 
configure, modify, and/or maintain. 

Further, a consolidated view of all customizations is basically Jira. If an institution is determining whether it needs to develop
a customization for a specific feature, enhancement, or bug, it currently lacks a portal to determine if a customization exists
or not or to what extent it needs to be updated.

Documentation for customizations are inconsistent. They can be in docx, Wiki, LaTeX, Docbook, or any number of other documentation 
formats. Mostly, the documentation is non-existent. Understanding what dependencies, configuration, or idiosyncracies that may exist
is difficult.

\section{A New Customization Methodology}

I propose a new methodology that streamlines the creation of customizations and provides conventions for managing the customizations.

\subsection{Built on Maven}

This methodology is built upon the concept of Maven dependency management which enforces that customizations can utilize transitive 
dependency management. Further, it will allow us to easily add customizations to an application. For example,

\begin{minted}[fontsize=\scriptsize]{xml}
<project xmlns="http://maven.apache.org/POM/4.0.0" xmlns:xsi="http://www.w3.org/2001/XMLSchema-instance"
	 xsi:schemaLocation="http://maven.apache.org/POM/4.0.0 http://maven.apache.org/xsd/maven-4.0.0.xsd">
  <modelVersion>4.0.0</modelVersion>
...
...
  <dependencies>
...
...
    <dependency>
      <groupId>edu.yourinstitution.kuali.kfs</groupId>
      <artifactId>customization1</artifactId>
      <version>1.0.0</version>
    </dependency>
  </dependencies>
...
...

</project>
\end{minted}

The dependency can contain the following:
\begin{itemize}
\item Liquibase changesets.
\item Spring configuration.
\item Other dependencies.
\item POM file(s).
\end{itemize}

\subsubsection{Directory Structure/Conventions}

\subsection{Custom Archetype for New Projects}
By using Maven for customizations, this means that there is a project that follows the conventions for Maven is required. To 
simplify the creation of customizations and also to ensure that the required conventions are followed, an archetype is provided. The
archetype includes:
\begin{itemize}
  \item Basic directory structure.
  \item Requirements for updating the Rice module to notify it of the customization.
  \item POM file for the customization.
  \item Initial Liquibase changelog for archetypes that use it.
\end{itemize}

The archetype is configurable based on different customization use cases. That is, if a customization requires a Spring change, the
necessary files are included to make that change. If there is a database change, an initial Liquibase changelog is provided to add
the change to. If there are java sources that will be included in the customization, sample integration and unit tests are provided.

\subsection{Private Nexus Repository}

A private repository will exist intended to be exclusively for Kuali application customizations. By being private, it will require
credentials to access simply by registering with the customization portal as a developer. Each developer gets his/her own 
credentials for accessing. Automated systems should use the credentials of the configuration manager.

\subsection{Integrated Database Change Management}

Liquibase has already been mentioned as the database change management tool. Its support as we as well as extensions will be available
as part of the proposed methodology.

Changelogs can take on many forms from XML
\begin{minted}[fontsize=\scriptsize]{xml}
<databaseChangeLog
    xmlns="http://www.liquibase.org/xml/ns/dbchangelog"
    xmlns:xsi="http://www.w3.org/2001/XMLSchema-instance"
    xmlns:ora="http://www.liquibase.org/xml/ns/dbchangelog-ext"
    xmlns:kualiext="http://www.liquibase.org/xml/ns/rice-lb-ext"
    xsi:schemaLocation="http://www.liquibase.org/xml/ns/dbchangelog 
    http://www.liquibase.org/xml/ns/dbchangelog/dbchangelog-3.2.xsd
	http://www.liquibase.org/xml/ns/dbchangelog-ext
    http://www.liquibase.org/xml/ns/dbchangelog/dbchangelog-ext.xsd
	http://www.liquibase.org/xml/ns/rice-lb-ext 
    http://r351574nc3.github.io/rice-lb-ext/rice-lb-ext.xsd">

    <changeSet id="create-type" author="lb-ext">
        <kualiext:createKimType namespace="KR-NS" 
                                  service="namespacePermissionTypeServiceTest" 
                                     name="Namespace">
	        <kualiext:attribute label="Test" 
                            namespace="KR-NS" 
                                 name="namespaceCode" 
                            component="org.kuali.rice.kim.bo.impl.KimAttributes" />
        </kualiext:createKimType>
    </changeSet>
</databaseChangeLog>
\end{minted}

or YAML

\begin{minted}[fontsize=\scriptsize]{yaml}
databaseChangeLog:
  - changeSet:
      id: create-type
      author: lb-ext
      changes:
        - createKimType:
            namespace: KRNS
            name: Namespace
              - attribute:
                  label: Test
                  namespace: KR-NS
                  name: namespaceCode
\end{minted}

\subsection{Integration with Kuali Jira}

A Jira plugin is currently in development that will allow jira issues for a given project to link to repositories, pull requests, 
and issues in GitHub. When a customization is released, Jira will automatically be updated with release information, and the dependency
code to add to your project for implementation

\begin{minted}[fontsize=\scriptsize]{xml}
<dependency>
    <groupId>edu.yourinstitution.kuali.kfs</groupId>
    <artifactId>customization1</artifactId>
    <version>1.0.0</version>
</dependency>
\end{minted}

\section{A Community Built on Github}

Github provides everything a project needs to get started quickly. 
\begin{itemize}
  \item APIs for a customization portal to gain access to pull request, issue and repository data surrounding a customization.
  \item Credentials, authentication and authorization management.
  \item Publicly available repositories to promote open source projects.
\end{itemize}

With the above it is possible for continuous integration to run upon modification. Releases can automatically promote nexus repository
updates. We can allow simple signups and logins simply by using Oauth and Github account information.

\subsection{Forking to Enhance Customizations}

Github provides easy forking for projects. If each customization is its own project, then forking is a simple button click in Github. 
Reasons for forking a project are few. Whatever customizations your institution uses, it is recommended that you fork it. Why? If 
there is an issue with the customization, a source code change will be required. How then do you handle that source code change? The
customization methodology would require the project to be forked. The change can then be made in the forked project and a pull
request can then be made to the original project. 

\subsection{Pull Requests to Improve Quality}
This is a normal pattern for handling changes that is intuitive to git users. As
a distributed VCS, git demands that each clone of a project is itself a repository. The best way to get changes into the original
source tree is to make a pull request. The pull request process allows for better vetting of code. Also, with the Github API, 3rd-party
applications like Continuous Integration tools and source code analysis tools can run against a pull request to verify it. This means,
that before source code even gets into the repository, we know that it has been tested and verified by an automated system and human
eyes.

\subsection{gh-pages, Issues, README, and Wiki for Documentation}

As part of the conventions, a Maven site is created. The Maven site is deployed to the Gihub gh-pages brance associated with the
repository for the customization. This will provide a front-facing documentation site for anyone that browses to the project page. 
The Maven site provides readers with information about dependencies of the customization, usage, Javadocs, source code cross-referencing,
 Github and Jira link information, and any other information the developer considered important to share with readers.

Further, Github supports markdown formated README and LICENSE files embedded within the repository to document the type of licensing
used for the customization as well as any documentation about how to use the customization.

Github also includes its own issue system which is helpful for managing defects and issues specifically related to the customization. 
Beyond that, there is also a wiki that can be used for documentation. A wiki may be a bit excessive for customizations, and it
in itself is not part of any convention enforced by the methodology. However, it is there if necessary.

\subsection{Consistent Rules and Guidelines for Quality Assurance}

As part of the release process flow, in order for a release to be successful, it must undergo automated testing that tests
the Kuali application itself and verifies its functionality. The application can be any application like Rice or KFS. 

\subsection{Provided Continuous Integration}

Continuous Integration runs against all branches whenever a change is made. Integration and unit tests included within the
customization project are executed as well as standard tests provided by the customization portal.

\subsection{Release Process Flow}

Upon pushing changes to the \textbf{release} branch
\begin{enumerate}
  \item Standard tests run in CI.
  \item Project tests run in CI.
  \item Maven release plugin executed.
\end{enumerate}

\section{True Continuous Delivery}

At this point, the best continuous delivery that can be offered for Kuali applications is that the application deploys continuously
between stops and restarts; however, true continuous delivery allows deployments without the application ever stopping. 

\subsection{Built on the New Customization Methodology}
There are several customizations that will be available through the customization portal that allow true continuous delivery. Each
customization offers a step closer.

\subsection{Live Changes to DataDictionary}
This customization abstracts the DataDictionary out of in-memory beans into a Redis NOSQL database. This allows the modifications to
be made to the DataDictionary without the application running which means that modifications also no longer require the application
to restart.

\subsection{Production Supported Hot Code Replacement}

In JDK 8, the JDI and JDPA allows better support for hot code replacement including full schema replacement of classes. This allows an 
entire class to be replaced if necessary within the running virtual machine. 

Prior to JDK 8, JDI and JDPA do support hot code replacement. It is not as extensive as in JDK 8, but it provides nominal replacement.

This feature does not exist yet and is not implemented as a customization since this feature is mostly configuration related and cannot
be described as a customization; however, it is something being investigated and will be supported in different implementations.

\subsection{Maven Dependency Management at Runtime Not Buildtime}

There is a customization project that embeds Maven within Kuali applications allowing the application to load dependencies at runtime
using a \verb|UrlJarClassloader| and reading dependencies from the project POM included in the WAR. With hot code replacement, classes and other 
classpath resources are loaded at runtime and can in most cases replace existing resources. For example, with JDK 8, a dependency
change will force transitive dependencies to be retrieved without rebuild or restart. 

One specific use case for this is when adding a new customization. Once the POM is updated with the customization, Maven will detect
the dependency change and begin retrieving dependencies.

\subsection{Liquibase Dependency Management}

Similar to the embedded Maven dependency management, there is a project for Liquibase changelog execution at runtime. Liquibase already
supports using Spring to execute changelogs upon startup; however, they are only executed once at startup. With this customization,
if a changelog is added to the project, it will be discovered and executed. 

One specific use case for this is when adding a new customization. Once the dependencies are retrieved for the customization, Liquibase
will detect a changelog within the classpath coming from that new customization and execute it.

\subsection{Alternative Templating and View Resolution}

This is a future planned project that replaces Freemarker with Mustache templating. Mustache templates are supported in various
languages like Scala, Java, Javascript, Ruby, C\#, Python, Perl, Clojure, Scheme, and many others. At this time, Freemarker template 
modifications require application restart to refresh changes. This project will install a new view resolver that will make use of realtime
Mustache rendering.

\section{Use Cases}

A list of use cases have been put together to give a more case-specific perspective of the features within the Customization Portal.

\subsection{Creating a New Generic Customization}

Execute the following Maven command:

\begin{minted}[fontsize=\scriptsize]{bash}
> mvn kfs:newCustomization -DgroupId=edu.yourinstitution.kuali.kfs \
                        -DartifactId=customization1 \
                           -Dversion=1.0-SNAPSHOT
\end{minted}

\subsection{Customization that Includes Only Changelogs}

Execute the following Maven command:

\begin{minted}[fontsize=\scriptsize]{bash}
> mvn kfs:newCustomization -DgroupId=edu.yourinstitution.kuali.kfs \
                        -DartifactId=customization1 \
                           -Dversion=1.0-SNAPSHOT \
                        -Dclassifier=changelogs
\end{minted}

\subsection{Customization Change to Templates}

\subsection{Releasing a Customization}

\subsection{Testing a Customization}

\section{What's Left}

\subsection{Portal User Interface}

\subsection{Archetype}

\subsection{Kuali Maven Integration}

\subsection{Better Hot Code Replacement Support}

\subsection{Mustache Templates to Replace Freemarker}

\section{Appendices}
\subsection{Structure}

\subsection{Artifacts}

\subsubsection{POM}

\subsubsection{javadoc}

\subsubsection{tests}

\subsubsection{classes}

\subsubsection{changelog}

\end{document}

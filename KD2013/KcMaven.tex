\documentclass[xcolor=dvipsnames,14pt,professionalfonts]{beamer}
\usepackage{minted}
\usepackage{etoolbox}
\usepackage[T1]{fontenc}
\usepackage{lmodern}
\usepackage[no-math]{fontspec} 
\usetheme{kualidays2013}
\usefonttheme{professionalfonts}
\usecolortheme{dolphin}

%\definecolor{foreground}{gray}{0}
%\definecolor{background}{gray}{1}
%\definecolor{keyword}{rgb}{0.2,0.2,0.8}
%\definecolor{warning}{rgb}{0.8,0.2,0.2}
%\definecolor{shadow}{gray}{0.35}
%\definecolor{hide}{gray}{0.9}
%\definecolor{figure}{rgb}{1,0.7,0}
%\definecolor{title}{rgb}{25,240,250}
\definecolor{title}{rgb}{0.09,0.84,0.88}
\definecolor{frametitle}{rgb}{0,0,0}
\definecolor{normal}{rgb}{0,0,0}

%\usecolortheme[named=keyword]{structure}

\setbeamercolor{title}{fg=title}
\setbeamercolor{frametitle}{fg=frametitle}
\setbeamercolor{section in toc}{fg=foreground}
\setbeamercolor{normal text}{bg=brown!46,fg=normal}

\setbeamerfont{structure}{family=\fontspec{Georgia},series=\bfseries} 
\setbeamerfont{subtitle}{family=\fontspec{Helvetica},series=\bfseries} 
\begin{document}

\title{Si Vis Bellum Para Bellum}
\subtitle{KC with Maven Overlays: Best Practices, Caveats, and  Anti-Patterns}
\author[Leo]{Leo Przybylski}

\usebackgroundtemplate%
{%
    \includegraphics[width=\paperwidth,height=\paperheight]{images/header.png}%
}

{
\usebackgroundtemplate{\includegraphics[width=\paperwidth]{images/title.png}}%
\begin{frame}[plain]
  \titlepage
\end{frame}
}

\begin{frame}{Overview}
  \begin{itemize}
  \item Welcome \& Colophon.
  \item About Presenters.
  \item Overlay Basics.
  \item Overlay Patterns.
  \item Overlays and VCS.
  \item Rice Modules.
  \item Overlays and Appserver Plugins.
  \item Rice DataDictionary Reloading vs. JRebel.
  \item Proposal: Contributing a Modification as a Maven Dependency.
  \item Multi-module vs. Regular Webapp Project.
  \end{itemize}
\end{frame}

\begin{frame}{Welcome and Colophon}
  \begin{itemize}
    \item Hi!
    \item ``Si Vis Bellum Para Bellum'' is a play on ``Si Vis Pacem
      Para Bellum'' and Web Archive (WAR) files. It means ``If you
      wish for war, prepare for war.'' WAR being a web archive rather than turning your project into a battlefield.
  \end{itemize}
\end{frame}

\begin{frame}{About Presenters}
\end{frame}

\begin{frame}{Overlay Basics}
  \begin{itemize}
  \item How Overlays work.
  \item What overlays are not.
  \item First Overlay.
  \end{itemize}
\end{frame}

\begin{frame}{How Overlays Work}
  \begin{itemize}
    \item A fileset or maven project copied over a webapp.
    \item Webapp does not have to be a maven project.
    \item The overlay does not have to be a complete webapp.
    \item An overlay does not even have to be very different at all.
  \end{itemize}
\end{frame}

\begin{frame}{What Overlays are Not}
  \begin{itemize}
  \item Project Inheritence.
  \item A replacement/alternative for multi-module projects.
  \item A silver-bullet for all webapp projects.
  \item Change management pattern.
  \end{itemize}
\end{frame}


\begin{frame}{Advantages of Overlays}
  \begin{itemize}
    \item Offers flexibility in your project by allowing granular
      modifications at the project level.
    \item Smaller projects for tracking changes with.
    \item Easier to modularize your project.
    \item Better code reuse.
  \end{itemize}
\end{frame}

\begin{frame}{First Overlay}
  Using KC as an Example to switch out CAS support for the login dummy
  \begin{enumerate}
    \item Package KC.
    \item Install as a prototype.
    \item Setup prototype jar.
    \item Create overlay project.
    \item Add CAS dependency.
    \item Add dependencies for KC.
    \item Configure war plugin for overlay.
  \end{enumerate}
\end{frame}

\begin{frame}[fragile]{Package and Install KC as a prototype}
  \begin{enumerate}
  \item Do install
  \begin{minted}[fontsize=\footnotesize]{sh}
mvn install
  \end{minted}
  \item Create a and Install Prototype Jar
  \begin{minted}[fontsize=\scriptsize]{sh}
mvn jar:jar
mvn install:install-file \
    -DgroupId=org.kuali.kra \
    -DarchetypeId=kc_project \ 
    -Dversion=5.1  \
    -Dpackaging=jar \
    -Dfile=target/kc_project-5.0.1.jar \
    -DpomFile=pom.xml
  \end{minted}
  ... or ...
  \begin{minted}[fontsize=\scriptsize]{sh}
mvn deploy:deploy-file \
    -DgroupId=org.kuali.kra \
    -DarchetypeId=kc_project \ 
    -Dversion=5.1  \
    -Dpackaging=jar \
    -Dfile=target/kc_project-5.0.1.jar \
    -DpomFile=pom.xml
  \end{minted}
\end{enumerate}
\end{frame}

\begin{frame}[fragile]{Create an Overlay Project the Old-Fashioned
    Way}
  \begin{itemize}
  \item Configure the maven-war-plugin
  \end{itemize}
  \begin{minted}[fontsize=\scriptsize]{xml}
<plugin>
  <groupId>org.apache.maven.plugins</groupId>
  <artifactId>maven-war-plugin</artifactId>
  <version>\${maven-war-plugin.version}</version>
  <configuration>
    <overlays>
      <overlay>
        <groupId>\${kcPrototypeGroupId}</groupId>
        <artifactId>\${kcPrototypeArtifactId}</artifactId>
        <excludes>
          <exclude>WEB-INF/classes/ApplicationResources.properties</exclude>
          <exclude>WEB-INF/tags/portal/</exclude>
          <exclude>WEB-INF/web.xml</exclude>
          <exclude>**/lib/**</exclude>
        </excludes>
      </overlay>
    </overlays>
  </configuration>
</plugin>
\end{minted}
\end{frame}

\begin{frame}[fragile]{Configure the maven-war-plugin}
  \begin{minted}[fontsize=\scriptsize]{xml}
<dependencies>
<!--
    These are the prototype dependencies. It is used by the overlay to refer back
    to the classes in the prototype.
  -->
  <dependency>
    <groupId>\${kcPrototypeGroupId}</groupId>
    <artifactId>\${kcPrototypeArtifactId}</artifactId>
    <version>\${kcPrototypeVersion}</version>
    <type>war</type>
  </dependency>
  
  <dependency>
    <groupId>\${kcPrototypeGroupId}</groupId>
    <artifactId>\${kcPrototypeArtifactId}</artifactId>
    <version>\${kcPrototypeVersion}</version>
    <type>jar</type>
  </dependency>
</dependencies>
  \end{minted}
\end{frame}

\begin{frame}[fragile]{Overlay Archetypes}
  \begin{itemize}
  \item Leo has developed an archetype for creating and overlay KC project.
  \item https://github.com/r351574nc3/kualigan-maven-plugins/tree/master/kc-maven-plugin
  \item To create an Overlay Project from Archetype
  \begin{minted}[fontsize=\scriptsize]{sh}
mvn -Dkc:create-overlay \
-DgroupId=com.rsmart.kuali.kra \
-DartifactId=kc -Dversion=5.1
  \end{minted}
  \end{itemize}
\end{frame}

\begin{frame}{Overlay Patterns}
  \begin{itemize}
    \item Overlaying configuration modifications.
    \item Overlay to activate a module.
    \item Overlay theming for UX/UI.
    \item Filtering overlay resources.
  \end{itemize}
\end{frame}

\begin{frame}[fragile]{Institutional Modification Best Practices}
  \begin{itemize}
    \item Use Rice Modules whenever possible.
    \item Use institution packages and namespaces instead of Kuali ones. For example,
      \begin{itemize}
        \item \texttt{org/kuali/kra}
        \item \texttt{org/kuali/rice}
      \end{itemize}
    \item Use \texttt{rice.kr.additionalSpringFiles}, et al to override spring configuration.
  \end{itemize}
\end{frame}

\begin{frame}[fragile]{Tips and Tricks}
  \begin{itemize}
    \item Use JRebel! It's free!
    \item Take advantage of the default classpath for runtime spring overrides.
    \item
  \end{itemize}
\end{frame}

\begin{frame}[fragile]{Proposal: Contributing a Modification as a Maven Dependency}
  \begin{itemize}
    \item Develop modifications as a separate maven project or module.
    \item Include the modifications as dependencies in your Maven Overlay project.
    \item Distribute modifications with documentation to a Maven Repository.
  \end{itemize}
\end{frame}

\begin{frame}[fragile]{Rice DataDictionary Reloading vs. JRebel.}
\end{frame}

\begin{frame}{Multi-module vs. Regular Webapp Project}
  \begin{itemize}
    \item Always difficult to build/manage multi-module webapp
      projects.
%    \item Jar dependencies from the parent may not match up;
%      therefore, will not get overlaid (redundant jars).
    \item Module artifacts are jars. This is adds problems for plugins that run on unassembled projects.
  \end{itemize}
\end{frame}

\begin{frame}{Solutions to Multi-module Issues}
  \begin{itemize}
    \item Build classes into \texttt{web\/target\/classes}
      \begin{itemize}
        \item Requires a profile to only do this for developers
        \item Not appropriate for production at all
      \end{itemize}
    \item JRebel will watch multiple src folders for changes
    \item Jetty will also watch multiple src folders
  \end{itemize}
\end{frame}

\begin{frame}[fragile]{Configuring JRebel}
  \begin{minted}[fontsize=\scriptsize]{xml}
<plugin>
    <groupId>org.zeroturnaround</groupId>
    <artifactId>jrebel-maven-plugin</artifactId>
    <configuration>
      <alwaysGenerate>true</alwaysGenerate>
      <addResourcesDirToRebelXml>true</addResourcesDirToRebelXml>
      <!--
        root is 2 directories away from jar/war modules
      -->
      <relativePath>../../</relativePath>
      <!--
        use a system property for specifing root directory (note the double $)
        start your application with -Drebel.root=c:/projects/
      -->
      <rootPath>${rebel.root}</rootPath>
    </configuration>
</plugin>
    \end{minted}
\end{frame}

\begin{frame}{JRebel and the KC Archetype}
\begin{itemize}
  \item The KC Archetype automatically configures your overlay as a multi-module project
  \item Comes with JRebel configuration already
\end{itemize}
\end{frame}

\begin{frame}{The end}
Thanks
\end{frame}
\end{document}

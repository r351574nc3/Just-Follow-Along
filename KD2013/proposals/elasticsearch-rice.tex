\documentclass[12pt,notitlepage]{article}
\author{Leo Przybylski}
\usepackage{fancyhdr}
\usepackage{graphicx}
\title{\includegraphics[width=0.75\textwidth]{kuali_base.png}\\ElasticSearch:
Documents and Objects Indexed and Searchable from Anywhere}
\date{}
\pagestyle{fancy}
\fancyhead{} % Clear all header fields 
\fancyhead[OL]{\sectionmark}
\fancyhead[OR]{\includegraphics[height=26pt]{kuali_base.png}}% 
\fancyhead[ER]{\sectionmark}
\fancyhead[EL]{\includegraphics[height=26pt]{kuali_base.png}}% 

\begin{document}
\maketitle
\abstract{ElasticSearch is a well-known tool that exposes Lucene
  indexing via JSON ReST API. It also manages the indexing of objects
  that are available. Therefore, it is capable of indexing business
  objects and documents native to Rice and particularly
  KEW. ElasticSearch then fronts this index with a JSON ReST API
  adding a unified solution giving access to Rice business objects. This
  session is about how to wire ElasticSearch to Rice and other Kuali applications, access the
  indexes, and use cases for this application.}

\section{Objectives}
\begin{itemize}
\item Developers will see how to connect Elastic Search.
\item Developers will learn/understand how to develop Rice
  applications that use ElasticSearch.
\item Developers will see example use cases for ElasticSearch and Rice.
\end{itemize}

\section{Audience}
Technical track for
\begin{itemize}
\item developers
\end{itemize}


\subsection{Level}
Advanced

\section{Focus Area}
Technical

\section{Track}
Kuali Rice

\section{Format}
Information Session

\end{document}